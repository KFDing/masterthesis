\documentclass[]{article}
\usepackage{graphicx}
\graphicspath{ {./images/} }
\usepackage{caption}
\usepackage{subcaption}
\usepackage{amsthm}
\usepackage{amsfonts}
\usepackage{amsmath}
\usepackage{amssymb}
\usepackage{mathrsfs}
\usepackage[ruled,linesnumbered]{algorithm2e}
\newtheorem{mydef}{Definition}[section]
\newtheorem{mytheorem}{Theorem}[section]

%opening
\title{Master Thesis --  Math Formalization}
\author{Kefang Ding} 
\date{9 Nov 2018}

\begin{document}

\maketitle
\hrulefill
\begin{abstract}
Model repair in process mining aims to improve existing process model according to actual event log. Event log is divided into positive and negative instances based on given KPIs. However, the current repair technologies use only positive instances, while negative instances are ignored. This results in a  less precision model. This article focuses on incorporating both positive and negative instances to repair model, in order to provide a model with better precision. 
\\
Firstly, a directly-follows graph is created from an existing process model in form of Petri net,  positive and negative instances of event log. Then, the directly-follows graph is transferred to a process tree, which is used to generate the final model in Petri net. To improve the precision of Petri net, long-term dependency is analyzed and added to Petri net. 
\\ 
In comparison to current technologies, the methods proposed in this articles provide better result with respect on precision in most cases. 
\end{abstract}
\pagebreak
\section{Introduction}
In this section, we give the problem, but extend the old writing. 
\\

Section description. The next sections are organized in the following order. Firstly, related work is given and compared. Secondly, our methods are introduced and formal definitions are given. In Implementation Section, the details of algorithms are given. Later, we evaluate our methods with simulated data and real data respectively and list the results. At last, a conclusion on the algorithms is provided. 

It is about 1 page!!

\section{Formal Definition}
Copy the existing document and paste here.. 

I did some introduction, it is proven that in some cases, we can't deal with it. When the $XORB\_S \neq LT\_S || XORB\_T \neq LT\_T$.

Another way to avoid this problem is to add duplicated events, but the problem stays the same, so if we want to keep the model fit, we add new event for the discovered long-term dependency, the original, we keep it in the model?? But it is not precise!!! It allows too many choices there, but the model is sound, because we add events on model, we produce and consume tokens from duplicated events, and consumes it later.. Still, it is not so right. But if we choose the events before, we decides the events later, it's true... helps a little... 

How to prove it ?? By induction. 
\begin{itemize}
	\item the original model is sound
	\item after adding one long-term dependency by duplicating the event, we make sure
	 \\ duplicated events are added into the xor branches, if it's chosen, then it consumes one token, at this xor branches, then it produces two tokens, one of which is put back again into the 
	 
	 
	 To prove the added places and duplicated events between two xor branches will not violate the soundness of model. 
	 \item 
	 \begin{itemize}
	 	\item adding duplicated events of one long-term dependency does not violate the soundness of model. 
	 	
	 	\item To connect the source and target  of long-term dependency which are the duplicated events will not violate the soundness.
	 	\\ One extra place is added to connect the source and target. After executing the source, one token is generated in this place; Due to long-term dependency, only this target is triggered by this token and it consumes this token. No extra token is introduced into this model, so the model keeps sound. 
	 	
	 \end{itemize}
	 
	 \item While keeping the original event in the model,  the model is with less precision.  
	 \\ 
	 So I want to solve this problem, and make it preciser by considering the original events into model... 
	 One drawback exists there, still... If we use the old and then the 
\end{itemize}

It's about 12 pages. But one problem, some basic information, we forget to give .. Maybe, we can put the related work later!!!


\section{Implementation}

We can give some screen shoots and describes the work here..

For the implementation, we need to consider the correctness of our methods.

More details here are around 6 pages
\section{Evaluation}
Some tests are given, one is simulated data, and also some examples are given to explain the advantages. Later, real data is listed here.. 

It's about 3 pages or 4 pages with graphs. 

\section{Related Work}
Related work, we give examples of new-generated model of Inductive Miner, Alpha Miner, <In which cases, it gives better results. > 
Then, repair model, 
Fahland's methods,  
Anna's methods,

To incorporate the negative information, simulated data are used, to limit the choices of going..

It is about 2 pages!!
\section{Conclusion}
One page is enough. 
%% 1 + 1-2 + 12 + 6-8 + 4-5 + 2-3 + 1 + 1-2 =  28-34 pages. It's ok..
\end{document}