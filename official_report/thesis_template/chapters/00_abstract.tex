\cleardoublepage

\begin{center}
%\paragraph{Kurzfassung}
%\hrulefill
%\end{center}
% Here goes the german abstract

\vspace {2cm}
\begin{center}
\paragraph{Abstract}
\hrulefill
\end{center}
\begin{flushleft}
Model repair in process mining aims to improve existing process model according to actual event log. Event log is divided into positive and negative instances based on given KPIs. However, the current repair technologies use only positive instances, while negative instances are ignored. This results in a  less precision model. This article focuses on  incorporating both positive and negative instances to repair model, in order to provide a model with better precision. 

Firstly, a directly-follows graph is created from an existing process model in form of Petri net,  positive and negative instances of event log. Then, the directly-follows graph is transferred to a process tree, which is used to generate the final model in Petri net. To improve the precision of Petri net, long-term dependency is analyzed and added to Petri net.

In comparison to current technologies, the methods proposed in this articles provide better result with respect on precision in most cases. 
\end{flushleft}