%% the performance factor is not considered in the reality.
%% In the order of: 
%% 1. Process mining (what it is and why relavant?? Because of the digital transformation??)
%% 2. Process Enhancement (what it is and wyh relavant)
%% 3. Current state of Process Enhancement 
%% 4. The contribution of this thesis
%% 5. Insights on the advantages, or finding of this method
Based on business execution history recorded in event logs, Process Mining provides visual insights on the business process and supports process analysis and enhancements. It bridges the gap between traditional business process management and advanced data analysis techniques such as data mining and gains more interests and application in recent years.

%% Process enhancement 
Process enhancement, as one of the main focuses in process mining, improves the existing processes according to actual business execution in the form of event logs. The records in an event log can be classified as positive and negative according to predefined Key Performance Indicators, e.g., the logistic time, and production cost in a manufacture. Most of the current enhancement techniques only consider positive instances from an event log to improve the model, while the value hidden in negative instances is simply neglected.

This thesis provides a novel strategy that considers not only the positive instances and the existing model but also incorporate negative information to enhance a business process. Those factors are balanced on directly-follows relations of activities and generate a process model. Subsequently, long-term dependencies of activities are detected and added to the model, in order to block negative instances and obtain a higher precision.

We validate the ability of our methods to incorporate negative information with synthetic data at first. The results showed that our method is able to overcome the shortcomings of current repair techniques and provide models with higher precision in given situations. Furthermore, we conducted experiments on real life data with the scientific workflow platform KNIME and verified feasibility of our proposed method in reality.