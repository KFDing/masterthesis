% what kind of conclusion do you want to get?? My method works in some cases while other methods can not. 
% <1> deal with some situations
% <2> improve precision and accuracy compared to existing model
% Further work:<1> consider multiple KPIs in the model and adjust them
%    <2> Add long-term dependency directly on Petri net
%    <3> deal with the unsound model and fix them in some way
%    <4> simply using the threshold to generate model, but want to integrate with machine learning and generate good model

% Also, some limitations about our methods, we need to point them out.
%% can not give out the best setting
%% it is absoulte values, not vary due to the problem.
In this thesis, we explore ways to use negative information in model repair and propose our innovative method. Firstly, we analyzed the current techniques on model repairs based on performance and detect their shortcomings. Then we  proposed a general framework to incorporate the forces from the existing model, positive and negative event logs. Three abstraction data models in the same type are built to represent those forces. Later, forces are balanced based on data models and expressed in a new data model. From this new data model, process models are discovered and converted into repaired models with the required type. Optional post processes include long-term dependency detection and silent transition reduction, which further improves the repaired model. 

Moreover, we demonstrate the usage of our method by conducting experiments with real life data. Our method is able to provide better repaired models than other repair methods in some situations. Also, with respect to other methods, it runs faster and generates simpler models. 

Future work might be to improve the rules of balancing different forces, which choose the directly-follows relation on the simple subtraction and sum of those forces. Advanced data mining techniques such as association rules discovery, and Inductive Logistic Programming can be used on those forces to derive rules for building a process model. The same improvement can be applied on the long-term dependency discovery. Moreover, in this implementation, the long-term dependency discovery is restricted on the activities with exclusive choices relation. Later, we should extend the long-term discovery on other possible relations, like parallel relation. Also, we can drop the process tree as our intermediate result and adopt it directly on the Petri net. 

