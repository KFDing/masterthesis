Real-life processes are rarely static over time. Deviations and performance problems may follow certain temporal patterns. Also, the places in a Petri net can behave very differently. Some places might always be stable, while others may exhibit more fluctuating behavior. To allow perspectives like conformance and performance to be fine grained enough, the analysis needs to be localized over multiple dimensions. In this approach, we first project the log on individual places, i.e. localizing over control flow. Then, the transition executions on a place are paired up with a heuristic, to extract complete and incomplete interactions. Since they have defined start and end times, they are further localized over time and associated to time intervals. For these intervals, we define metrics for conformance, performance and busyness, as a form of local process context. This allows a much finer analysis because averages tend to obscure lots of valuable information. Additionally, we provide a dataset made up of all interactions together with the localized metrics for their duration.

To evaluate our approach, we applied it comparatively to a purpose-generated synthetic log and a real-life log. We showed that timeseries of metrics can be of great help when diagnosing conformance and performance. Especially compared to the standard alignment-based plugins, the negligible additional computing time makes using our plugin worthwhile. A drawback however, is that we do not support activity execution times based on lifecycle information of the events. This perspective is still invaluable for a thorough performance analysis.

As future work, the approach could be parallelized further to reduce computation time. It could even be implemented in a map-reduce fashion for distributed computing. As it is, scalability is limited by the usage of alignments which we noted in Section \ref{sec:performance}. To remedy this, a different way of handling silent and duplicate transitions could be explored or results of decomposition could be applied, since the approach is localized to individual places anyway.
Another drawback of our implementation is that it only supports case attributes and not event attributes which are even more appropriate for such a localized concept. And lastly, an encompassing analysis of the exportable dataset could be performed. Especially concept drift detection and clustering may be applied here.