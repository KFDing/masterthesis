%% we should write the inductive miner , directly-follows relation in this paper,
%% but how to organize them together??
To update an existing process model in organizations, there are two strategies, rediscovery and process enhancement. Process rediscovery applies the discovery techniques on the actual event log to mine a new model. Process enhancement improves the model based on not only the actual event log but also the existing model. 


Process discovery has been intensively researched in the past two decades and many algorithms have been proposed\cite{van2016data}. Directly-follows\cite{van2004workflow, leemans2013discovering} methods investigates the activities order in the traces and extracts higher relations which are used to build process models. State-based methods like \cite{bergenthum2007process, cortadella1995synthesizing}  builds a transition system to describe the event log, and then group the state regions into corresponding Petri net node. Language-based algorithms uses integer linear system to represent the place constraint where the token at one place can never go negative. By solving the system, a petri net is created. Its representative techniques are Integer Linear Programming(ILP) Miner\cite{van2008process}. Other methods due to \cite{van2009process} include search-based algorithm like Genetic Algorithm Miner\cite{de2007genetic}, heuristic-based algorithm Heuristics Miner\cite{weijters2003rediscovering}.

Among those discovery methods, Inductive Miner is widely applied\cite{leemans2013discovering}. It investigates the activity order in the traces and represents the order in a directly-follows graph. Based on the graph, it finds the most prominent split from the set of exclusive choice, sequence, parallelism and loop splits on the event log.  Afterwards, the corresponding operator to the split is used to build a block-structured process model called process tree. Iteratively, the  split sublogs are passed as inputs for the same procedure until single activity is reached and no split is available. A process tree is output as the mined process model.

When the actual event log differs a lot from the referred process model, it is suitable to use the rediscovery method to improve the business execution. However, in some cases, the process enhancement focuses to extend or improve an existing process model by using an actual event log\cite{van2011process}. Besides extending the model with more data perspectives, repair is another type of enhancement. It modifies the model to reflect observed behavior while keeping the model as similar as possible to the original model.

In \cite{fahland2012repairing}, model repair is firstly introduced into process enhancement. By using conformance checking, the deviations of the event log and process model are detected. The consecutive deviations in log only are collected in the form of subtraces at specific location Q in the model. Later, the subtraces are grouped into sublog that share the same location Q for subprocess discovery. In the earlier version in \cite{fahland2012repairing}, the sublogs are obtained in a greedy way, while in \cite{fahland2015model}, sublogs are gathered by using ILP Miner to guarantee the fitness. Additional subprocesses and loops are introduced into the existing model to ensure the fitness, which also brings variants of execution paths into the model. 

Later, compared to \cite{fahland2012repairing, fahland2015model}, where all deviations are incorporated in model repair, \cite{dees2017enhancing} considers the impact of negative information.  In \cite{dees2017enhancing}, the deviations of the model and event log are firstly analyzed, in order to find out which deviations enforces the positive performance. Given a trace and a selected KPI, an observation instance is built to correlate the number of each log move with KPI output. Based on the observation instance,  a set of rules are derived in the form of a decision tree. According to the rules, the original event log is divided into sublogs with traces matching the rules. The sublogs are then repaired to contain only trace deviations which have a positive KPI output. Following repair, the sublogs are merged as the input for model repair in \cite{fahland2015model}. According to the study case in \cite{dees2017enhancing}, it provides better result than \cite{fahland2015model} on the aspect of performance. 
 
As described above, the state-of-the-art repair techniques are based on positive instances, meanwhile the negative information are neglected. Without negative information, it is difficult to balance the fitness and precision of those model. Likewise, few researches give a try to incorporate negative information in multiple forms on process discovery.

In \cite{goedertier2009robust}, the negative information is artificially generated by analyzing the available events set before and after one position and represented in the form of the complement of positive event sets. Based on the positive and negative event sets, Inductive Logic Programming is applied to detect the preconditions for each activity. Those preconditions are then converted to Petri net after applying a pruning and post-process step. Similar work on model discovery based on artificial negative events are published later. In \cite{vanden2014determining}, the author improves the method in \cite{goedertier2009robust} by assigning weights on artificial events with respect to unmatching window, in order to offer generalization on model. 

The work in \cite{ponce2016incorporating} uses traces in the event log with negative outcomes as negative information. It extends the techniques of numerical abstract domains and Satisfiability Modulo Theories(SMT) proposed in \cite{carmona2014process} to incorporate negative information for model discovery. Each trace  as positive or negative is transformed as one point in n-dimensional space, n is the number of distinct activities. The execution of a trace reflects the token transmission and marking limits on places in the model. Those limits are represented into the a set of marking inequalities and in a form of convex polyhedron in n-dimensional space. Given half-space hypotheses, SMT solves the inequalities and gives the limits on the process model. Before SMT, negative information is incorporated to shift and rotate the polyhedron, which limits the generalization of the solution space. Because half-space is used, this method can not deal with negative instances overlapped into positive instances.


However, the field of model repair which considers the negative information is new. Furthermore, the idea to incorporate negative instances on trace level into model repair is innovative.  

\iffalse
Compared to this, our approach is innovative mainly in the following aspects. 
\begin{itemize}
	\item Incorporate the negative information into model repair. 
	\item Analyze the long-term dependency in the model to provide a preciser result. 
	\item Analyze Model on Trace level. All activities constituting a trace are considered. 
\end{itemize}
\fi