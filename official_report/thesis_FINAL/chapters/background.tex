%% we should write the inductive miner , directly-follows relation in this paper,
%% but how to organize them together??
To update an existing process model in organizations, there are two strategies, rediscovery and process enhancement. Process rediscovery applies the discovery techniques on the actual event log to mine a new model. Process enhancement improves the model based on not only the actual event log but also the existing model. 


Process discovery has been intensively researched in the past two decades and many algorithms have been proposed \cite{van2016data}. Directly-follows methods \cite{van2004workflow, leemans2013discovering} investigate the  order of activities in the traces and extract higher relations which are used subsequently to build process models. State-based methods like  \cite{bergenthum2007process, cortadella1995synthesizing}  build a transition system to describe the event log, and then extract Petri net places from the related state regions. Language-based algorithms use integer linear system to represent the place constraint where the token at one place can never go negative. By solving the system, a petri net is created. Its representative techniques are Integer Linear Programming (ILP) Miner \cite{van2008process}. Other methods due to  \cite{van2009process} include search-based algorithms, like Genetic Algorithm Miner \cite{de2007genetic}, and heuristic-based algorithms, like Heuristics Miner \cite{weijters2003rediscovering}. Among those discovery methods, Inductive Miner bases on the directly-follows relation and is widely applied \cite{leemans2013discovering} in practice. 

%It investigates the activity order in the traces and represents the order in a directly-follows graph. Based on the graph, it finds the most prominent split from the set of exclusive choice, sequence, parallelism, and loop splits on the event log.  Afterward, the corresponding operator to the split is used to build a block-structured process model called a process tree. Iteratively, the  split sublogs are passed as inputs for the same procedure until one single activity is reached and no split is available. The mined process model is a process tree that can be converted into the classical process model Petri net. 
% write something about the conformance checking
Conformance checking analyzes the deviations between a reference process model and observed behaviors driven from its execution.  To find commonalities and discrepancies, token-based replay starts from the initial place of the Petri net model and fires the corresponding activities in the event log. If the final place is reached without any missing tokens which are added to enable activities or any remaining tokens that are produced by the firing of activities, the trace is fitting without any deviations. Alignment-based conformance checking consists of moves which represent the  matching situations between the transitions of a  model and the events in event log. By minimizing the costs with different types of moves, it finds the optimal alignment between the trace and the model and detect the discrepancies.

When the actual event log differs a lot from the referred process model, it is suitable to use the rediscovery method to improve the business execution. However, in some cases, the process enhancement focuses to extend or improve an existing process model by using an actual event log \cite{van2011process}. Besides extending the model with more data perspectives, model repair is another type of enhancement. It modifies the model to reflect observed behavior while keeping the model as similar as possible to the original one \cite{fahland2012repairing}.

In  \cite{fahland2012repairing}, model repair is firstly introduced into process enhancement. By using conformance checking, the deviations of the event log and process model are detected. The consecutive deviations in log  are only collected in the form of subtraces at a specific location Q in the model. Subsequently, the subtraces are grouped into sublogs that share the same location Q for subprocess discovery. In the earlier version in  \cite{fahland2012repairing}, the sublogs are obtained in a greedy way, while in  \cite{fahland2015model}, sublogs are gathered by using ILP Miner to guarantee the fitness. Additional subprocesses are introduced into the existing model to ensure that all traces fit the model, but it introduces more behavior into the model and lowers the precision.

Later, compared to  \cite{fahland2012repairing, fahland2015model}, where all deviations are incorporated in model repair,  \cite{dees2017enhancing} considers the impact of negative information.  In  \cite{dees2017enhancing}, the deviations of the model and event log are firstly analyzed, in order to find out which deviations enforce the positive performance. Given a trace and a selected KPI, an observation instance is built to correlate  activities in event logs which are deviating from models. Based on the observation instance,  a set of rules are derived in the form of a decision tree. According to the rules, the original event log is divided into sublogs with traces matching the rules. The sublogs are then repaired with trace deviations which lead to positive KPI outputs. Following repair, the sublogs are merged as the input for model repair in  \cite{fahland2015model}. According to the study case in  \cite{dees2017enhancing}, \cite{dees2017enhancing} provides a better result than  \cite{fahland2015model} on the aspect of performance. 
 
As described above, the state-of-the-art repair techniques are based on positive instances, meanwhile, the negative information is neglected. This causes difficulty to balance the fitness and precision of those models. Likewise, little research gives a try to incorporate negative information in multiple forms on process discovery.

In  \cite{goedertier2009robust}, the negative information is artificially generated by analyzing the available event set before and after one position and represented in the form of the complement of positive event sets. Based on the positive and negative event sets, Inductive Logic Programming is applied to detect the preconditions for each activity. Those preconditions are then converted to Petri net after applying a pruning and post-process step. In  \cite{vanden2014determining}, the author improves the method in  \cite{goedertier2009robust} by assigning weights on artificial events with respect to an unmatching window, in order to offer generalization on the model. 

The work in  \cite{ponce2016incorporating} uses traces in the event log with negative outcomes as negative information. It extends the techniques of numerical abstract domains and Satisfiability Modulo Theories (SMT) proposed in  \cite{carmona2014process} to incorporate negative information for model discovery. Each trace  as positive or negative is transformed as one point in n-dimensional space, n is the number of distinct activities. By adding places between transitions, it limits the transition execution by demanding tokens on those added places. This limit can be reflected by the sequence of traces.  In the linear system, those limits are represented into a set of inequalities and in a form of a convex polyhedron in n-dimensional space. Given half-space hypotheses, SMT solves the inequalities and gives the limits on the process model. Before SMT, negative information is incorporated to shift and rotate the polyhedron, which limits the generalization of the solution space. Because half-space is used, this method can not deal with negative instances overlapped on positive instances.


However, the field of model repair which considers the negative information is new. Furthermore, the idea to incorporate negative instances on trace level into model repair is innovative.  

\iffalse
Compared to this, our approach is innovative mainly in the following aspects. 
\begin{itemize}
	\item Incorporate the negative information into model repair. 
	\item Analyze the long-term dependency in the model to provide a preciser result. 
	\item Analyze Model on Trace level. All activities constituting a trace are considered. 
\end{itemize}
\fi