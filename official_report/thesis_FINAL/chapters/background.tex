Process enhancement focuses to extend or improve an existing process model by using actual event log\cite{van2011process}. Besides extending the model with more data perspectives, repair is another type of enhancement. It modifies the model to reflect observed behavior while keeping the model as similar as possible to original model.

In \cite{fahland2012repairing}, model repair is firstly introduced into process enhancement. By using conformance checking, the deviations of alignment are detected. The consecutive log moves is collected in the form of subtraces at specific location Q. Later, the subtraces are grouped into sublog that share the same location Q for subprocess discovery. In the earlier version in \cite{fahland2012repairing}, the sublogs are obtained in a greedy way, while in \cite{fahland2015model}, sublogs are gathered by using ILP miner to guarantee the fitness. Additional Subprocesses and loops are introduced into the existing model to ensure the fitness, which also brings variants of execution paths into the model. 

Later, compared to \cite{fahland2012repairing, fahland2015model}, where all deviations are incorporated in model repair, \cite{dees2017enhancing} considers model performance into model repair. An observation instance is built to represent the type of log moves given trace and  KPI. Subsequently, a classification tree will be constructed from the set of observation instances to cluster traces of event log. Then, the techniques in \cite{fahland2015model} are applied into event log with positive traces to repair model. 


As described above, the state-of-the-art repair techniques are based only on positive instances, meanwhile the negative instances are neglected. Without negative instances, it is difficult to balance the fitness and precision of those model. Few researches give a try on incorporating negative information into process discovery. \cite{goedertier2009robust}  analyzes the available events set before and after one position and generates artificial negative events based on the complement of those event sets. Next, Inductive Logic Programming is applied to detect the preconditions for each activity. Those preconditions are then converted to petri net after applying a pruning and post-process step. 

Similar work on model discovery based on negative information are published later. In \cite{vanden2014determining}, the author improves the method by assigning weights on artificial events with respect to unmatching window, in order to offer generalization on model. 

\cite{ponce2016incorporating} extends the techniques of numerical abstract domains and Satisfiability Modulo Theories(SMT) used in \cite{carmona2014process} to incorporate negative information for model discovery. Each trace in the log is treated as positive or negative and later transformed as one point in n-dimensional space, n is the number of distinct activities. The execution of a trace reflects the token transmission and marking limits on places in the model. Those limits are represented into the a set of marking inequalities and in a form of convex polyhedron in n-dimensional space. Given half-space hypotheses, SMT solves the inequalities and gives the limits on the process model. Before SMT, negative information is incorporated to shift and rotate the polyhedron, which limits the generalization of the solution space. Because half-space is used, this method can not deal with negative instances overlapped into positive instances.

However, the field of model repair which considers the negative information is new. 

To incorporate the negative information, simulated data are used, to limit the choices of going..

Compared to this, our approach is innovative mainly in the following aspects. 
\begin{itemize}
	\item Incorporate the negative information into model repair. Unlike the methods mentioned before
	\item Analyze the long-term dependency in the model to provide a preciser result. 
	\item Analyze Model on Trace level. All activities constituting a trace are considered. 
\end{itemize}