Process aware information systems (PAIS) are becoming more and more widespread. These information systems log huge amounts of events and even more event data can be extracted from all other kinds of ubiquitous organizational databases. Even this complex task is becoming easier and better supported with current research. Additionally, due to the current Big Data craze, many other types of data sets are in need of analysis.

To conduct such analyses the L* lifecycle model for process mining \cite{vanderAalst2011} can be employed. It is composed of five steps. At first the project obviously needs to be planned and justified. Then, event data is extracted and analysis goals like improvement of certain KPIs (key performance indicators) are determined. As the third step, control flow models are discovered, compared and reasoned about. Starting with this model a great variety of different approaches can be applied to gain insights about other perspectives like time, data flow, resources and process context.
Finally, but this may only be possible for very structured processes (also called lasagna processes), operational support can be derived from the analyzed historical data and applied to active cases.

Analyzing the different perspectives can be done at different granularities/dimensionalities. Without considering another dimension one could get one value for fitness, like the average of all traces. O