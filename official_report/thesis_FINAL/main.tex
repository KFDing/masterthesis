\documentclass[a4paper,11pt,twoside,openright]{WCarticle}


\renewcommand{\gTitle}{Model Repair by Incorporating Negative Instances In Process Enhancement}
\renewcommand{\gAuthor}{Kefang Ding}
\renewcommand{\gHeader}{Master Thesis}
\renewcommand{\gSupervisor}{Prof. Wil M.P. van der Aalst\\
Prof. Thomas Rose\\
Dr. ir. Sebastiaan J. van Zelst}
\renewcommand{\gInstitution}{PADS RWTH University}

\begin{document}
%%% Title page
\pagenumbering{roman}

\tpage
\pagestyle{plain}
%%% TOC

\chapter*{Acknowledgments}\addcontentsline{toc}{chapter}{Aknowledgement}
The acknowledgments and the people to thank go here, don't forget to include your project advice. 

%\begin{abstract}
\chapter*{Abstract} \addcontentsline{toc}{chapter}{Abstract}
Process enhancement is one of the three main focus in process mining. Model repair, as an important part of process enhancement, improves an existing process model by fixing the deviation between the model and the actual execution recorded into an event log. Meanwhile, it aims to keep the new generated model as similar as possible with the existing one. 

During the repair, 
However, few of the contributions on model repair take into account of negative information, while negative instances according to certain KPIs are ignored. 

Few researches incorporates the negative information from event log on model repair. This article focuses on  incorporating both positive and negative instances to repair model, in order to provide a model with better precision. 

Firstly, directly-follows relation of activities are analyzed respectively from an existing process model in form of Petri net,  positive and negative instances of the event log. Then, a directly-follows graph is built from those directly-follows relations and subsequently transferred into a process tree and Petri net. Based on the process tree, long-term dependency is analyzed and added on the final model in Petri net.

In comparison to current technologies, the methods proposed in this articles provide better result with respect on precision in most cases. 
%\end{abstract}

\tableofcontents
\listoffigures
\listoftables
%\setcounter{page}{1}
\cleardoublepage
%%% Begin document
\pagestyle{fancy}

\chapter{Introduction} \label{chap:intro}\pagenumbering{arabic}
%\documentclass[../main.tex]{subfiles}
%\begin{document}
Process mining is a relatively new discipline that has emerged from the need to bridge the gap of data mining and business process management. The objective of process mining is to support the analysis of business process, provide valuable insights on processes and further improve the business execution. According to \cite{van2011process}, techniques of process mining are divided into three categories: process discovery, conformance checking and process enhancement. Process discovery techniques focus on deriving process models from event logs of the information system, allowing the vision into the real business process. Conformance checking analyzes the deviations between an referenced process model and observed behaviors driven from its execution. Enhancement adapts and improves existing process models by extending the model with additional data perspectives or repairing the existing model to accurately reflect observed behaviors. 

Due to the increasing availability of detailed event logs of information systems, process mining techniques have recently enabled wider applications of process mining in organizations around the world\cite{van2011process}.  After applying process discovery in organizations, a process model is fixed in information system to guide the execution of business. However,in real life, business processes often encounter exceptional situations where it is necessary to execute process differing from the predefined model. To reflect the reality, the organizations need to adapt the existing process model. Basically, one can apply process discovery techniques again to obtain a new model from event log. However, due to the facts, (1) the cost of rediscovery, and (2)  the discovered model tend to have less similarity with the original model\cite{fahland2012repairing}. As shown in \cite{fahland2012repairing}, there is a need to change an existing model similar to the original model while replaying the current process execution. Here comes the model repair. 

Model repair belongs to process enhancement and stands between process discovery and conformance checking. It analyzes the workflow deviations between event log  and process model, and fix the deviations mainly by adding sub processes on the model. As known, business in organizations is goal-oriented and aims to have high performance according to a set of Key Performance Indicators(KPIs), for example, average conversion time for the sales, payment error rate for the finance. However, there are few researches on applying the process mining with consideration of performance\cite{ghasemi2016process}.  \cite{ghasemi2016process} points out the rare contributions like \cite{dees2017enhancing} to combine performance into process mining. Deviations are firstly analyzed to determine if they have a positive impact on the process performance. Model repair techniques in \cite{fahland2015model} are applied into traces with positive deviations.

However, the current repair methods have some limits. Model repair fixes the model by adding subprocesses, silent transitions or loops, it guarantees the model fitness but overgeneralizes the model, such that it allows more behaviors than expected. On the other hand, it increases the model complexity.  Even the performance is considered in \cite{dees2017enhancing}, but only deviations in positive is used to add subprocesses, the negative information is ignored, which disables the possibility to block negative behaviors from model.  A motivation example is listed to describe those limits.
%% Motivation 
\section{Motivation Example}
The example is extracted from the registration procedure of thesis project one German university. The main activities are topic selection, make proposal, a meeting with supervisor to discuss the topic, and check course requirements. After finishing all of those activities, the formal registration is enabled and the procedure comes to end.
Figure \ref{fig:model_a} shows the original process in Petri net. The activities are modeled by the corresponding \emph{transitions} which is represented by a square. Transitions are connected through a circle called \emph{place}. Transitions and places build the static structure of Petri net. Tokens in black dots are put in places and represent the 

and flow through the network to model the possible states. Tokens 

As an example, there is a model presented in Figure \ref{fig:model_a}, where A is followed directly by B. During its execution in real life, an event log is generated: 
\[{ <A, B> }^{55}  , {<B, A>}^{105} \] 
When considering KPIs performance, the log is divided into a positive and a negative set: 
\[ Positive \;  examples: { <A, B> }^{5}  , {<B, A>}^{100} \] 
\[ Negative \; examples: { <A, B> }^{50}  , {<B, A>}^{5} \]  
After applying current model repair techniques in \cite{fahland2015model}, the process model is repaired using all examples. A can be skipped and duplicated later in a self-loop. In \cite{dees2017enhancing} methods, only the positive examples are taken for the model repair. Yet, since both $<A,B> and <B,A>$ contributes to good performance,  the repaired model keeps the same like in Figure \ref{fig:model_examples}(b).
\begin{figure}[h]
	\centering
	\begin{subfigure}[b]{\textwidth}
		\centering
		\includegraphics[width=\linewidth]{figures/introduction/Thesis-PN-Example-original-model.jpg}
		\caption{original process model}
		\label{fig:model_a}
	\end{subfigure}
	\hfill
	\begin{subfigure}[b]{\textwidth}
		\centering
		\includegraphics[width=\linewidth]{figures/introduction/Thesis-PN-Example-model-with-fitness.jpg}
		\caption{process model \\ with high fitness}
		\label{fig:model_b}
	\end{subfigure}
	\hfill
	\begin{subfigure}[b]{\textwidth}
		\centering
		\includegraphics[width=01.0\linewidth]{figures/introduction/Thesis-PN-Example-model-with-lt-1.jpg}
		\caption{process model \\ with good KPI}
		\label{fig:model_c}
	\end{subfigure}
	
	\caption{example for model change under model repair}
	\label{fig:model_examples}
\end{figure}

However, it's obvious that $<A,B>$ often leads to bad performance and therefore should be excluded. The Figure \ref{fig:model_examples}(c) shows the expected model with incorporating the negative information. This model reinforces positive examples and avoids negative instances, which provides us a more accurate view of the business process.

Clearly, the use of negative information can bring significant benefits, e.g, enable a controlled generalization of a process model: the patterns to generalize should never include negative instances. The demand to improve current repair model techniques with incorporating negative instances appears. In the next section, the demand is analyzed and defined in a formal way.

\section{Research Problem And Questions}
We analyze the current model repair methods, and give the formal definitions.
\begin{definition}
Given an input of one existing process model M, an event log L and a set of KPIs, how  to improve current process enhancement techniques by incorporating negative information, and generate a process model to enforce the positive instances while blocking the negative instance, with condition that the generated model should be as similar to the original model as possible Therefore, the repaired model provides a better way to understand and execute the real business process compared to the original model.
\end{definition}

%% here we should list the picture of our method, we can't really use methodology 

This paper tries to provide a solution for it. Our idea is to analyze the positive and negative impact on process performance of each trace. It balances the existing model, positive traces and negative traces on directly-follows relation, in order to incorporate all the factors on model generation. Later, the directly-follows relation is used to create process model by Inductive Miner. What's more, the impact of the existing model, positive and negative instances are parameterized by weights, to allow more flexibility of the generated model.

\section{Outline}
The reminder is organized in the following order. Section 2 recalls the basic notions on process mining and list the preliminary to solve the problem. The next section lists our methods are introduced and formal definitions are given. In Implementation Section, the details of algorithms are given. Later, we evaluate our methods with simulated data and real data respectively and list the results. Subsequently, the discussion on this paper is presented. At last section, a conclusion is drawn on the paper. 


%\end{document}


\chapter{Background} \label{chap:backgrd}
%% we should write the inductive miner , directly-follows relation in this paper,
%% but how to organize them together??
To update an existing process model in organizations, there are two strategies, rediscovery and process enhancement. Process rediscovery applies the discovery techniques again on the actual event log. Process enhancement improves the model based on not only the actual event log but also the existing model. 

Process discovery has been intensively researched in the past two decades and many algorithms have been proposed. They can be classified into the following categories, based on directly-follows relation, state-based regions, language-based regions are others. Directly-follows methods investigates the activities order in the traces and extracts higher relation which are used to build process models. State-based methods like builds a transition system to describe the event log, and then group the state regions into corresponding petri net node. Language-based algorithms uses integer linear system to represent the place constraint where the token at one place can never go negative. By solving the system, a petri net is created. Its representative techniques are Integer Linear Programming(ILP) Miner\cite{van2008process}. 

Alpha Miner, Inductive Miner, Integer Linear Programming(ILP) Miner. Alpha Miner scans traces in event log and finds the directly-follows relation of two activities in the trace. Then, higher relation like sequence, parallel and no directly-follows are driven according to the combination of directly-follows relation and used to create footprint matrix. The matrix is at last converted into Petri net. 

Inductive Miner guarantees to generate sound process models. A directly-follows graph are built according to the event log. It finds the most prominent split in event log, after the detecting the operators which include exclusive choice, sequence, parallelism and loop, the operators are used to build the process tree. Sublogs are created due to this operator and as inputs for the same procedure until single activities. Process tree is a block-structured model and can be easily transformed into Petri net.  

State-base regions

Language-based regions
 
Process enhancement focuses to extend or improve an existing process model by using an actual event log\cite{van2011process}. Besides extending the model with more data perspectives, repair is another type of enhancement. It modifies the model to reflect observed behavior while keeping the model as similar as possible to original model.

In \cite{fahland2012repairing}, model repair is firstly introduced into process enhancement. By using conformance checking, the deviations of alignment are detected. The consecutive log moves is collected in the form of subtraces at specific location Q. Later, the subtraces are grouped into sublog that share the same location Q for subprocess discovery. In the earlier version in \cite{fahland2012repairing}, the sublogs are obtained in a greedy way, while in \cite{fahland2015model}, sublogs are gathered by using ILP miner to guarantee the fitness. Additional subprocesses and loops are introduced into the existing model to ensure the fitness, which also brings variants of execution paths into the model. 

Later, compared to \cite{fahland2012repairing, fahland2015model}, where all deviations are incorporated in model repair, \cite{dees2017enhancing} considers model performance into model repair. An observation instance is built to represent the type of log moves given trace and  KPI. Subsequently, a classification tree will be constructed from the set of observation instances to cluster traces of event log. Then, the techniques in \cite{fahland2015model} are applied into event log with positive traces to repair model. 


As described above, the state-of-the-art repair techniques are based only on positive instances, meanwhile the negative instances are neglected. Without negative instances, it is difficult to balance the fitness and precision of those model. Few researches give a try on incorporating negative information into process discovery. \cite{goedertier2009robust}  analyzes the available events set before and after one position and generates artificial negative events based on the complement of those event sets. Next, Inductive Logic Programming is applied to detect the preconditions for each activity. Those preconditions are then converted to petri net after applying a pruning and post-process step. 

Similar work on model discovery based on negative information are published later. In \cite{vanden2014determining}, the author improves the method by assigning weights on artificial events with respect to unmatching window, in order to offer generalization on model. 

\cite{ponce2016incorporating} extends the techniques of numerical abstract domains and Satisfiability Modulo Theories(SMT) used in \cite{carmona2014process} to incorporate negative information for model discovery. Each trace in the log is treated as positive or negative and later transformed as one point in n-dimensional space, n is the number of distinct activities. The execution of a trace reflects the token transmission and marking limits on places in the model. Those limits are represented into the a set of marking inequalities and in a form of convex polyhedron in n-dimensional space. Given half-space hypotheses, SMT solves the inequalities and gives the limits on the process model. Before SMT, negative information is incorporated to shift and rotate the polyhedron, which limits the generalization of the solution space. Because half-space is used, this method can not deal with negative instances overlapped into positive instances.

However, the field of model repair which considers the negative information is new. 

To incorporate the negative information, simulated data are used, to limit the choices of going..

Compared to this, our approach is innovative mainly in the following aspects. 
\begin{itemize}
	\item Incorporate the negative information into model repair. Unlike the methods mentioned before
	\item Analyze the long-term dependency in the model to provide a preciser result. 
	\item Analyze Model on Trace level. All activities constituting a trace are considered. 
\end{itemize}

\iffalse
\chapter{Preliminaries} \label{chap:prelim}
This chapter uses definitions adapted from \cite{van2013process, hompes2015detecting, baier2015matching} to give a background of the formalization of event logs, Petri nets and alignments. First off, some basic mathematical foundations are required.

\begin{definition}[Tuples]
For sets $X_1, X_2, \ldots, X_n$, $t \in X_1 \times X_2 \times \ldots \times X_n$ is an $n$-tuple. For $1\le i\le n$, $t_i$ is the projection on the $i$th element in the tuple.
\end{definition}

\begin{definition}[Functions]
For sets $X$ and $Y$, $f:X\to Y$ is a total function and $g: X\nrightarrow Y$ is a partial function. The domain of a function is the set of elements it is defined on, so $dom(f) = X$ and $dom(g) \subseteq X$ because a partial function does not have to be defined on every element. All elements in the domain are mapped to an element in the target set $Y$.
\end{definition}

\begin{definition}[Powerset]
For a set $X$, $\mathcal{P}(X)=\set{S}{S \subseteq X}$ denotes the set of all subsets of $X$.
\end{definition}

\begin{definition}[Multiset]
For a set $X$, a multiset $B:X\to \mathbb{N}_0$ over $X$ is a set with duplicates. $B(x)$ gives the frequency of $x$ in $B$. Multisets use square brackets e.g. $B_1=[a,a,b,c,c,c,c]=[a^2,b,c^4], B_2=[a,d]$. They overload the typical set operations $B_1 \uplus B_2 = [a^3,b,c^4,d], B_1 \cap B_2 = [a]$ and $B_1 \setminus B_2 = [a,b,c^4]$. The set of all multisets over $X$ is $\mathcal{B}(X)$.
\end{definition}

\begin{definition}[Sequences]
For any set $X$, $\sigma = \langle \sigma_1, \sigma_2, \ldots, \sigma_n \rangle \in X^*$ is a finite sequence over $X$ of length $|\sigma|=n$. $set(\sigma) = \set{x\in X}{\exists 1 \le i \le |\sigma|: \sigma_i = x}$ is the set of distinct elements in the sequence $\sigma$.
Two sequences $\sigma, \sigma^\prime \in X^*$ can be concatenated to $\sigma \cdot \sigma^\prime$, e.g. $\langle a,b \rangle \cdot \langle c \rangle = \langle a, b, c \rangle$.
Any (partial) function $f:X \nrightarrow Y$ to some set $Y$ can be applied to a sequence over $X$ recursively as follows. $f(\langle \rangle) = \langle \rangle$ and for $x\in X$ and $\sigma \in X^*$: 
$$f(\langle x \rangle \cdot \sigma) = \begin{cases}
f(\sigma) & \text{if } x\notin dom(f)\\
\langle f(x) \rangle \cdot f(\sigma) & \text{if } x \in dom(f)
\end{cases}$$
A sequence over $X$ can also be projected onto a subset $Q\subseteq X$ recursively. So $\langle \rangle \restriction_Q = \langle \rangle$ and for $x\in X$ and $\sigma \in X^*$: $$(\langle x \rangle \cdot \sigma)\restriction_Q = \begin{cases}
\sigma\restriction_Q & \text{if } x\notin Q\\
\langle x \rangle \cdot \sigma\restriction_Q & \text{if } x \in Q
\end{cases}$$
Sequences over tuples $\sigma \in (X_1 \times \ldots \times X_m)^*$ can be projected to a sequence over $X_i$ ($1\le i\le m$) with $\sigma \restriction_i$.
\end{definition}

\begin{definition}[Logic]
For a set $X$ and a formula $\phi:X\to \mathcal{B}$, $\forall x\in X: \phi(x)$ requires $\phi$ to hold for all $x$, $\exists x\in X: \phi(x)$ requires $\phi$ to hold for at least one $x$ and $\exists!\ x\in X : \phi(x)$ requires $\phi$ to hold for exactly one $x$. For two formulas $\phi$ and $\psi$, $\phi \wedge \psi$ requires both to hold, $\phi \vee \psi$ requires at least one to hold and $\phi \XOR \psi$ requires exactly one of them to hold.
\end{definition}

The recording of events during process execution is the basis of all process mining activities. Recorded events are always associated with an activity and a time. Other attributes are optional.
\begin{definition}[Event]
Let $\E$ denote the universe of all possible events. The set $ATE$ contains all possible event attribute names. For an event $e \in \mathcal{E}$ and attribute name $n\in ATE$, $n(e)$ is the value of attribute $n$ of event $e$. If $e$ does not have attribute $n$, $n(e)=\bot$. $\Univ A$ denotes the universe of all activities. The event attributes $activity:\mathcal{E}\to \Univ A$ and $time: \E \to Time$ are always defined.
\end{definition}
For readability, we write events together with their activity in the superscript like $e^{activity(e)}$.
Traces are a chronologically ordered sequence of such events and belong to a case along with possible further case attributes. Since events are considered to be unique, there cannot be any duplicates in a trace.
\begin{definition}[Trace]
A trace $\sigma = \langle e_1, e_2, \ldots, e_n \rangle \in \mathcal{E}^*$ is a finite sequence of events without duplicates, i.e. $\forall 1\le i < j\le n: e_i \neq e_j$. The events are non-strictly ordered by their time attribute, i.e. $\forall 1\le i < j\le n: time(e_i) \le time(e_j) $.
\end{definition}
Like an event, a case is a unique object with attributes, one of which being the associated trace.
\begin{definition}[Case]
Let $\C$ denote the universe of all possible cases. The set $ATT$ contains all possible case attribute names. For a case $c\in \C$ and attribute name $n \in ATT$, $n(c)$ is the value of attribute $n$. If $c$ does not have attribute $n$, $n(c) = \bot$. The attribute $trace(c)=\hat{c}\in \mathcal{E}^*$ is always defined. $\hat{c}$ will be used as a shorthand for the associated trace of a case.
\end{definition}
An event log is a collection of cases.
\begin{definition}[Event log]
Let $\Log \subseteq \C$. $\Log$ is an event log if every event in the log is unique, i.e. $\forall c,c^\prime \in \Log: c\neq c^\prime \implies set(\hat{c}) \cap set(\hat{c}^\prime) = \emptyset$. The universe of event logs is $\Univ L$.
\end{definition}

As a notation for process models, we use (labeled) Petri nets. Observe that, since different process notations (like BPMN models) can be converted between each other, this is not a restriction.
\begin{definition}[Labeled Petri net]
A (labeled) Petri net is a tuple $PN=(P,T,F,l)$ consisting of the set of places $P$, set of transitions $T$ with $P \cap T = \emptyset$, a flow relation $F\subseteq (P \times T) \cup (T \times P)$ and a labeling function $l:T \to \Univ A \cup \{ \tau \}$, with $\tau \notin \Univ A$. The labeling function maps each visible transition to an activity in $\Univ A$. Transitions $t\in T$ mapped to $\tau$ are called silent or invisible. They are unobservable. The sets $T^l=\set{t\in T}{l(t)\neq \tau}$ and $T^{\tau}=\set{t\in T}{l(t)=\tau}$ partition the transitions into those labeled with an activity and the silent ones.
\end{definition}
To be able to refer to adjacent nodes more easily, the preset and postset are defined. For places they contain transitions and for transitions places.
\begin{definition}[Preset, postset]
Let $PN=(P,T,F,l)$ be a Petri net. For every node $n\in P \cup T$, $n^\bullet=\{m | (n, m) \in F\}$ is the postset and $^{\bullet}n=\{m | (m, n) \in F\}$ the preset of $n$.
\end{definition}
A marking describes the state of a Petri net, represented by a distribution of tokens on the places.
\begin{definition}[Marking]
Let $PN=(P,T,F,l)$ be a Petri net. A marking $M\in \mathcal{B}(P)$ is a multiset of places. For every $p\in P$, $M(p)$ is the number of tokens on place $p$. A transition $t\in T$ is enabled in a marking $M$, written $M[t \rangle$, if each place in $\pre t$ has at least one token. The transition execution $M[t \rangle M^\prime$ results in the marking $M^\prime = (M \setminus \pre t) \uplus \post t$ where one token is removed from every place in the preset and one is added to every place in the postset.
\end{definition}
To make a Petri net executable, an initial marking which describes the initial distribution of tokens and a final marking which describes a valid completion states are required.
\begin{definition}[System net]
A system net $SN=(PN,M_i,M_f)$ is a Petri net $PN=(P,T,F,l)$ with an initial marking $M_i$ and final marking $M_f$. A valid firing sequence on $SN$ is a sequence of transitions $\sigma \in T^*$ such that $\forall 1\le j \le |\sigma| : M_{j-1}[\sigma_j \rangle M_{j}$ with $M_0=M_i$ and $M_{|\sigma|}=M_f$ where every transition execution was enabled.
\end{definition}

\begin{figure}
    \centering
    \includegraphics[width=.6\textwidth]{figures/preliminaries/simplenet.png}
    \caption{A simple Petri net}
    \label{fig:simplenet}
\end{figure}

\begin{figure}
    \centering
    $\gamma_1=\bigalignment{4}{$e_{11}$ & $e_{12}$ & $e_{13}$ & \doublegg}{b & a & c & }{\doublegg & a & c & b}{ & t1 & t3 & t4}$ and $\gamma_2=\bigalignment{3}{$e_{21}$ & \doublegg & $e_{22}$}{a & & b}{a & $\tau$ & b}{t1 & t2 & t4}$
    \caption{Two example alignments for $\hat{c}_1 = \langle e_{11}^b, e_{12}^a, e_{13}^c \rangle$ and $\hat{c}_2 = \langle e_{21}^a, e_{22}^b \rangle$}
    \label{fig:simplealign}
\end{figure}
For replaying a (possibly unfitting) trace on a system net which might contain silent transitions or multiple transitions with the same label (duplicate transitions) alignments from \cite{van2012replaying} are being used.
\begin{definition}[Alignment]
Let $SN=(PN, M_i, M_f)$ be a system net and $c\in \C$ a case. An alignment $\gamma \in (\E \cup \{\gg\}) \times(T \cup \{ \gg \}) \setminus \{(\gg,\gg)\}$ is a sequence of so-called moves such that $\gamma\restriction_1\restriction_{\E} = \hat c$ and $\gamma\restriction_2\restriction_T$ is a valid firing sequence on $SN$. And for every $1\le i \le |\gamma|$:
\begin{itemize}
    \item $\gamma_i = (e,t)$ with $e\in \E$, $t\in T$ is called a move in both and if $activity(e) = l(t)$ it is called a synchronous move,
    \item $\gamma_i = (e,\gg)$ with $e\in \E$ is called a log move,
    \item $\gamma_i = (\gg, t)$ with $t\in T$ is called a model move and if $l(t) = \tau$ it is called an invisible move.
\end{itemize}
Since there may be multiple valid alignments, a cost function is used to rank them and return the optimal one. The standard cost function, which is often used, gives sync moves and invisible moves cost $0$ and move on model/log cost $1$. A move on both which is not a synchronous move gets a cost of $\infty$ since it is usually not wanted. For the cost of the whole alignment, the costs for all moves are summed up.
\end{definition}
Figure \ref{fig:simplealign} shows two example alignments for the traces $\hat{c}_1 = \langle e_{11}^b, e_{12}^a, e_{13}^c \rangle$ and $\hat{c}_2 = \langle e_{21}^a, e_{22}^b \rangle$ on the model given in Figure \ref{fig:simplenet}. The top row contains events together with their $activity$ and the bottom row transitions together with their label to make it more readable. The alignments provide a path through the model even for unfitting traces like $\hat{c}_1$ or traces which require the execution of invisible transitions like $\hat{c}_2$.

\chapter{Approach} \label{chap:appr}
Analyzing conformance and performance expressed as averages over the whole model and log is very broad. Many real-life processes contain complex patterns and behavior as shown in the introduction. They can only be discovered when combining multiple perspectives on the data. The approach presented here focuses on localizing metrics for conformance, performance and process context to individual places in the Petri net model. The metrics are further localized to time intervals to make changes over time visible.

The approach consists of three steps:
\begin{enumerate}
    \item The log is projected onto the places of the process model to extract a \emph{locally mapped log}
    \item \emph{Interactions} are extracted from this \emph{locally mapped log}
    \item Metrics are calculated from \emph{interactions} for time intervals
\end{enumerate}

In the following, we introduce these steps in order.

\section{Locally mapping the log}
The first step requires some pre-processing. To properly handle the start and end places in the next steps, we insert unique start and end events into every trace. Their $activity$ is $\tstart$ or $\tend$ respectively. Their $time$ attribute is the time of first or last event respectively. Let $\E^\prime \supset \E$ be the universe of events including all artificial start and end marker events. To add the corresponding $\tstart$ and $\tend$ transitions in the model, a system net $SN=(PN,M_i,M_f)$ with initial and final marking is required. The $\tstart$ transition $t_{\tstart}$ is connected to all places in $M_i$ and the $\tend$ transition $t_{\tend}$ from all places in $M_f$. So for $PN = (P,T,F,l)$ the pre-processed Petri net is $PN^\prime=(P,T^\prime,F^\prime,l^\prime)$ with $F^\prime = F \cup \set{ (t_{\tstart},p) }{p\in M_i}\cup \set{(p,t_{\tend})}{p \in M_f}$, $T^\prime = T \cup \{t_{\tstart},t_{\tend}\}$, $\forall t\in T: l^\prime(t)=l(t)$ and $l^\prime(t_{\tstart}) = \tstart, l^\prime(t_{\tend})) = \tend$. Figure \ref{fig:preprocessednet} shows the pre-processed version of Figure \ref{fig:simplenet}.

\begin{figure}
    \centering
    \includegraphics[width=0.6\textwidth]{figures/concept/preproc_simplenet.png}
    \caption{The simple model after pre-processing}
    \label{fig:preprocessednet}
\end{figure}

From this point on, we assume all Petri nets and logs to be pre-processed in this way.
We partition the adjacent transitions of a place into visible and invisible transitions, so for a place $p\in P$, $adj(p)=(\pre p \cup \post p) \cap T^{l}$ is the set of adjacent visible transitions and $adj_{\tau}(p) =(\pre p \cup \post p) \cap T^{\tau}$ is the set of adjacent silent transitions. To be able to add silent transition executions, let $l_{\tau}:T^{\tau} \to A_{\tau}$ be a labeling function which assigns a unique artificial activity to every otherwise unlabeled $\tau$-transition. To further define the silent execution events, let $\I$ be the universe of all artificial silent (hence not in $\E^\prime$) events whose $activity$ attribute is a silent activity. With these definitions, a locally mapped case can be defined. It is a mapping of events from a case to transitions adjacent to a particular place.
\begin{definition}[Locally mapped case]
Let $\Log \in \Univ L$ be a log and $PN = (P,T,F,l)$ be a Petri net. For a case $c\in \Log$ and place $p\in P$, a locally mapped case $lmc$ for $p$ inherits (and overloads) all attributes of $c$ except for the $trace$ attribute. Its trace is overloaded to a sequence of pairs of real events from $\hat{c}$ and inserted artificial silent events together with a fitting transition, i.e. $\hat{lmc}=trace(lmc)\in ((\E^\prime \times adj(p)) \cup (\I \times adj_{\tau}(p)))^*$. For every pair $(e,t)\in \E^\prime \times adj(p)$ or $(i,t^\prime)\in \I \times adj_{\tau}(p)$ in the sequence, the label of the transition has to match the activity of the event, i.e. $activity(e) = l(t)$ and $activity(i) = l_{\tau}(t^\prime)$. All events in the sequence have to be unique and ordered by $time$, so $\forall 1\le i<j \le |\hat{lmc}|, \hat{lmc}_i=(e,t), \hat{lmc}_j = (e^\prime,t^\prime): e\neq e^\prime \wedge time(e) \le time(e^\prime)$. Lastly, the projection onto real events is a subset of the original trace, i.e. $set(\hat{lmc}\restriction_1\restriction_{\E^\prime}) \subseteq set(\hat{c})$. If $|\hat{lmc}| = 0$, $lmc$ is called an empty locally mapped case which means the case did not interact with the place $p$.
\end{definition}
There are many ways to map a case onto the locally mapped cases, so we first define a generalized strategy. The strategy also determines the mandatory $time$ attribute of inserted silent events.
\begin{definition}[Case mapping strategy]
Let $\Log \in \Univ L$ be a log and $PN = (P,T,F,l)$ a Petri net. For any case $c\in \Log$, a case mapping strategy $\func{case\_map\_strat(c)} = (lmc_p)_{p \in P}$ creates a locally mapped case for each place $p\in P$. It needs to be consistent because a transition execution mapped in one of the locally mapped cases needs to be included in all the others also adjacent to the transition. So, $\forall p\in P \ \forall (e,t) \in set(\hat{lmc}_p) \ \forall p^\prime \in (\pre t \cup \post t): (e,t)\in set(\hat{lmc}_{p^\prime})$.
\end{definition}
Consistency can be achieved by first selecting a subset of events, mapping them to transitions, adding silent executions and then projecting them onto the adjacent transitions of the places.
Now, the entire event log can be mapped onto individual places with locally mapped logs.
\begin{definition}[Locally mapped log]
Let $\Log \in \Univ L$ be a log, $PN = (P,T,F,l)$ be a Petri net and $\func{map\_strat}$ be a case mapping strategy. Then a locally mapped log $\Log_p$ of place $p\in P$ is the set of all non-empty locally mapped cases, i.e. $\Log_p=\set{lmc}{lmc = \func{map\_strat(c)}_p \wedge |\hat{lmc}|\neq 0 \wedge c \in \Log }$.
\end{definition}

We propose a specific case mapping strategy $\func{sync\_strat}$ where a global execution is mapped locally. It uses alignments to be able to handle duplicate transitions and silent transition executions. 
The synchronous moves already provide a mapping of a subset of events from the trace to transitions in the model, so only the silent transition executions have to be added. This is done by replaying the synchronous and invisible moves on the model in a token-based manner. Synchronous moves are always executed, even if not enabled. Invisible moves are only executed if they are enabled, to be able to set their $time$ attribute which is the instant they are enabled. That means chains of silent events consistently shift the sojourn duration between two real events to the place at the end of the chain. The invisible move chains can be cut-off by non-invisible model moves, leaving the deviation of missing this event on the places adjacent to these non-invisible model moves.
This way, the locally mapped cases can be seen as a projection of the fitting parts of the event log onto locally mapped logs for each place.
A variant $\func{all\_strat}$ also tries to consider log moves which makes it unsuitable for models with duplicate transitions. It essentially treats log moves which can be mapped to a transition as synchronous moves, executing them unconditionally. More information from the log is used that way. Especially loops in a trace which cannot be mimicked by the model and hence end up being log moves can contribute to the analysis this way.

\begin{algorithm}[tb]
\begin{algorithmic}
\Require given $PN = (P,T,F,l)$
\State $\forall p\in P: lmc_p \gets$ new locally mapped case
\State $\forall p\in P: \hat{lmc}_p \gets \langle \rangle$
\State $M \gets$ new empty marking
\State $\var{lastMappedTime}: P \nrightarrow Time$
\Function{mapStrat}{alignment $\gamma$, boolean $\var{mapLogMoves}$}
    \For{$i = 1 : |\gamma|$}
        \If{$\gamma_i = (e,t) \wedge activity(a) = l(t)$} \Comment{sync move}
            \State \Call{handleTransition}{$e$, $t$}
        \ElsIf{$\gamma_i = (e, \gg)$} \Comment{log move}
            \If{$\var{mapLogMoves} \wedge \exists t\in T: l(t) = activity(e)$}
                \State \Call{handleTransition}{$e$, $t$}
            \EndIf
        \ElsIf{$\gamma_i = (\gg, t) \wedge l(t) = \tau$} \Comment{inv move}
            \If{$M[t\rangle$}
                \State $enabledAt \gets \max\{ \var{lastMappedTime}(p)| p\in \pre t \}$
                \State $e_{\tau} \gets$ new unique event
                \State $activity(e_{\tau}) \gets l_{\tau}(t)$
                \State $time(e_{\tau}) \gets enabledAt$
               \State \Call{handleTransition}{$e_{\tau}$, $t$}
            \EndIf
        \EndIf
    \EndFor
    \State \Return $(lmc_p)_{p\in P}$
\EndFunction
\Function{handleTransition}{event $e$, transition $t$}
    \State $M \gets (M \setminus \pre t) \uplus \post t$
    \ForAll{$p \in {\pre t \cup \post t}$}
        \State $\hat{lmc}_p \gets \hat{lmc}_p \cdot (e,t)$
        \If{$p \in \post t$}
            \State $\var{lastMappedTime}(p) \gets time(e)$    
        \EndIf
    \EndFor
\EndFunction
\end{algorithmic}
\caption{The proposed $\func{map\_strat}$}
\label{alg:mapstrat}
\end{algorithm}

A configurable algorithm for both variants is presented in Algorithm \ref{alg:mapstrat}. Provided an alignment $\gamma$ of a case, it steps through the sequence of moves and updates the locally mapped cases accordingly in the helper function \textsc{handleTransition}. It uses the partial function $\var{lastMappedTime}$ to track timestamps and a marking $M$ as an updated current marking. The transitions of synchronous moves and, if desired, mapped log moves are fired regardless if they are enabled. Invisible model moves are executed if they were actually enabled otherwise they are ignored. When they are executed, a unique new event is created and its mandatory attributes set. Then it is treated like any other transition execution.

As an example, consider the pre-processed log cases $\hat{c_1} = \langle e_{11}^\tstart, e_{12}^b, e_{13}^a, e_{14}^c, e_{15}^\tend \rangle$ and $\hat{c_2} = \langle e_{21}^\tstart, e_{22}^a, e_{23}^b, e_{24}^\tend \rangle \}$ in log $\Log = \{ c_1, c_2\}$ on the model given in Figure \ref{fig:preprocessednet}. The results of both strategies on this log are shown in Figure \ref{fig:mapstrats}. With $\func{sync\_strat}$, the unfitting event $e_{12}^b$ of the first case is not mapped which assigns that deviation to places $p3$ and $p4$. $\func{all\_strat}$ maps this event, which hides the problem from $p4$ and focuses it on $p3$ as swapped events.
\begin{figure}
    \centering
    \begin{tabular}{|l|c|}
    \hline
    Local log & $\func{sync\_strat}$  \\
    \hline
    $\Log_{p1}$ 
    & $\{ \langle (e_{11}^\tstart,t5),(e_{13}^a,t1) \rangle, \langle (e_{21}^\tstart,t5), (e_{22}^a,t1) \rangle \}$ 
     \\
    $\Log_{p2}$ 
    & $\{ \langle (e_{13}^a,t1), (e_{14}^c,t3) \rangle, \langle (e_{22}^a,t1),(e_{\tau}^{t2},t2) \rangle \}$ 
    \\
    $\Log_{p3}$ 
    & $\{ \langle (e_{14}^c,t3) \rangle, \langle (e_{\tau}^{t2},t2),(e_{23}^b, t4) \rangle \}$ 
    \\
    $\Log_{p4}$ 
    & $\{ \langle (e_{15}^\tend, t6) \rangle, \langle (e_{23}^b,t4),(e_{24}^\tend, t6) \rangle \}$ 
    \\
    \hline
    \hline
    Local log & $\func{all\_strat}$ \\
    \hline
    $\Log_{p1}$ & $\{ \langle (e_{11}^\tstart,t5),(e_{13}^a,t1) \rangle, \langle (e_{21}^\tstart,t5), (e_{22},a) \rangle \}$ \\
    $\Log_{p2}$ & $\{ \langle (e_{13}^a,t1), (e_{14}^c,t3) \rangle, \langle (e_{22}^a,t1),(e_{\tau}^{t2},t2) \rangle \}$ \\
    $\Log_{p3}$ & $\{ \langle (e_{12}^b,t4),(e_{14}^c,t3) \rangle, \langle (e_{\tau}^{t2},t2), (e_{23}^b,t4) \rangle \}$ \\
    $\Log_{p4}$ & $\{ \langle (e_{12}^b,t4),(e_{15}^\tend,t6) \rangle, \langle (e_{23}^b,t4),(e_{24}^\tend,t6) \rangle \}$ \\
    \hline
    \end{tabular}
    \caption{Comparison of local log mapping strategies}
    \label{fig:mapstrats}
\end{figure}

\section{Extracting Interactions}
The next step relies on the locally mapped logs to pair up input events with output events. During token-based replay, the execution of an input activity of a place will produce a token on it and the execution of an output activity of the place will eventually consume the token. This is a complete interaction. The time difference between token production and consumption on the place is called sojourn time. If the trace is not fitting perfectly, this results in places where a token will be missing or remaining. These cases are classified as incomplete interactions.
\begin{definition}[Complete interaction, incomplete interaction]
Let $PN = (P,T,F,l)$ be a Petri net and $\hat{lmc} = \langle \hat{lmc}_1, \hat{lmc}_2,\ldots, \hat{lmc}_n \rangle$ be the trace of a locally mapped case of $p\in P$. A complete interaction is a tuple $ci = ( (e,t), (e^\prime,t^\prime) )$ where $(e,t) = \hat{lmc}_i, (e^\prime, t^\prime) = \hat{lmc}_j$ such that $1\le i < j \le n$ and $t \in{\pre p}$ and $t^\prime \in{\post p}$. All interactions have a $start$, $end$ and $duration$ attribute. $start(ci) = time(e)$, $end(ci) = time(e^\prime)$ and $duration(ci) = end(ci) - start(ci)$. An incomplete interaction $ii$ is a tuple where the missing pair is replaced by $\missup$ or $\missdown$. So with $(e,t) \in set(\hat{lmc})$: $ii = (\missup, (e,t))$ for $t\in \post p$, $ii = ((e,t), \missdown)$ for $t\in \pre p$ and any of those for $t\in \pre p \cap \post p$. That means self loops can appear in two different incomplete interactions. The attributes are $start(ii)=end(ii)=time(e)$ and $duration(ii) = 0$. Additionally, interactions also further inherit (and overload) the case attributes of the locally mapped case.
\end{definition}
To extract these interactions in a sensible way, we define a generic interaction extraction strategy. This heuristic further maps the locally mapped cases to all contained interactions.
\begin{definition}[Interaction sets, interaction extraction strategy]
Let $PN = (P,T,F,l)$ be a Petri net and $lmc$ a locally mapped case of $p\in P$. $\CIS$ is called a complete interaction set and $\IIS$ an incomplete interaction set. An interaction extraction strategy $\func{extract\_strat_p(lmc)}=\var{(\CIS, \IIS)}$ for place $p$ has to partition all pairs $(e,t)\in set(\hat{lmc})$ into valid complete and incomplete interactions. Pairs with a self loop transition have to appear in two interactions because they first consume a token then produce one again. So, for all $1\le j \le  |\hat{lmc}|=n$ with $\hat{lmc}_j=(e,t)$ the following has to hold:
\begin{align}
    &t\in \pre p \implies (\exists!\ j < k \le n: (\hat{lmc}_j, \hat{lmc}_k)\in \CIS) \XOR ((\hat{lmc}_j,\missdown) \in \IIS) \label{eq:1}\\
    &t\in \post p \implies (\exists!\ 1\le i < j: (\hat{lmc}_i, \hat{lmc}_j)\in \CIS) \XOR ((\missup,\hat{lmc}_j) \in \IIS) \label{eq:2}
\end{align}
\end{definition}
 In other words, a self loop either is in two complete interactions, first as an output transition second as an input transition. Or just in one of those complete interactions and in one incomplete one, or both the input and the output transition part of incomplete interactions.
 We propose a basic algorithm for an $\func{extract\_strat}$ with Algorithm \ref{alg:extractstrat}. It performs a local token-based replay on the place where events which try to consume non-existent tokens and events which produce tokens that are never consumed are classified as incomplete interactions. It is also configurable by choosing the internal data structure which stores the producing events as either a stack or a queue. This determines the heuristic used to determine which overlapping events belong together. A small example illustrating the difference is given in Figure \ref{fig:stackqueue}. Complete Interactions are double headed arrows and incomplete ones are \texttimes. The queue matches the first producer with the first consumer while the stack matches the last producer with the first consumer.
\begin{algorithm}
\begin{algorithmic}
\Require given $PN=(P,T,F)$ and place $p\in P$
\Function{extractInteractions}{locally mapped case $lmc$}
    \State $\CIS \gets \{\}$
    \State $\IIS \gets \{\}$
    \State $\var{upTimes} \gets new(stack)$ \Comment{or $new(queue)$}
    \For{$i = 1 : |\hat{lmc}|$}
        \State $(e,t) \gets \hat{lmc}_i$
        \If{$t \in \post p$}
            \If{$empty(\var{upTimes})$}
                \State $\IIS \gets \IIS \cup (\missup,(e,t))$
            \Else
                \State $(e^\prime, t^\prime) \gets remove(\var{upTimes})$
                \State $\CIS \gets \CIS \cup ((e^\prime, t^\prime), (e,t))$
            \EndIf
        \EndIf
        \If{$t \in{\pre p}$}
            \State $add(upTimes, (e,t))$
        \EndIf
    \EndFor
    \If{not $empty(\var{upTimes})$}
        \ForAll{$(e,t) \in \var{upTimes}$}
            \State $\IIS \gets \IIS \cup ((e,t), \missdown)$
        \EndFor
    \EndIf
    \State \Return $(\CIS, \IIS)$
\EndFunction
\end{algorithmic}
\caption{The proposed $\func{extract\_strat}$}
\label{alg:extractstrat}
\end{algorithm}
\begin{figure}
    \centering
    \includegraphics[scale=1]{figures/concept/overlap_strategy.pdf}
    \caption{A comparison of the queue and stack interaction extraction}
    \label{fig:stackqueue}
\end{figure}

Finally, the interaction sets for the local logs in Figure \ref{fig:mapstrats} of the running example are given in Figure \ref{fig:interactionstrats}. As mentioned before, $\func{sync\_strat}$ spreads out the incomplete interactions to all adjacent places while $\func{all\_strat}$ focuses it on the input places. Since swapped events can lead to two incomplete interactions, mapping log moves may lead to more incomplete interactions. In this example, using a stack or queue makes no difference.

\begin{figure}
    \centering
    \begin{tabular}{|l|c|c|}
    \hline
    \multirow{2}{*}{Local log} & \multicolumn{2}{|c|}{$\func{sync\_strat}$}  \\
     & $\CIS$ & $\IIS$ \\
    \hline
    $\Log_{p1}$ 
    & $\{ ((e_{11}^\tstart,t5),(e_{13}^a,t1)), ((e_{21}^\tstart,t5), (e_{22}^a,t1)) \}$ & $\{\}$ 
     \\
    $\Log_{p2}$ 
    & $\{ ((e_{13}^a,t1), (e_{14}^c,t3)), ((e_{22}^a,t1),(e_{\tau}^{t2},t2)) \}$ & $\{\}$ 
     \\
    $\Log_{p3}$ 
    & $\{ ((e_{\tau}^{t2},t2),(e_{23}^b, t4)) \}$ & $\{ ((e_{14}^c,t3),\missdown) \}$ 
     \\
    $\Log_{p4}$ 
    & $\{ ((e_{23}^b,t4),(e_{24}^\tend,t6)) \}$ & $\{ (\missup,(e_{15}^\tend,t6)) \}$
     \\
    \hline
    \hline
    \multirow{2}{*}{Local log} & \multicolumn{2}{|c|}{$\func{all\_strat}$} \\
    & $\CIS$ & $\IIS$ \\
    \hline
    $\Log_{p1}$ & $\{ ((e_{11}^\tstart,t5),(e_{13}^a,t1)), ((e_{21}^\tstart,t5), (e_{22}^a,t1)) \}$ & $\{\}$ \\
    $\Log_{p2}$ & $\{ ((e_{13}^a,t1), (e_{14}^c,t3)), ((e_{22}^a,t1),(e_{\tau}^{t2},t2)) \}$ & $\{\}$ \\
    $\Log_{p3}$ & $\{ ((e_{\tau}^{t2},t2),(e_{23}^b, t4)) \}$ & $\{ (\missup,(e_{23}^b,t4)), ((e_{14}^c, t3),\missdown) \}$ \\
    $\Log_{p4}$ & $\{ ((e_{12}^b,t4), (e_{15}^\tend,t6)), ((e_{23}^b,t4),(e_{24}^\tend,t6)) \}$ & $\{\}$ \\
    \hline
    \end{tabular}
    \caption{Continued running example with interaction sets}
    \label{fig:interactionstrats}
\end{figure}

\section{Time intervals} \label{timeintervals}
Until now, we extracted interactions of the log with places. This only uses the control-flow dimension and ordering of events. Since interactions also have the attributes $start$, $end$ and $duration$, it is possible to order them and spread them out over the time dimension. Complete interactions may also have a non-zero duration, so it is not trivial to filter them into time intervals. There are different types of overlap of an interaction with a time interval.
\begin{figure}
    \centering
    \includegraphics[scale=2]{figures/concept/timeinterval_overlap.pdf}
    \caption{Different types of overlap of interactions with a time interval}
    \label{fig:interactionoverlap}
\end{figure}
\begin{definition}[Time interval projections]
A time interval $[I_{start}, I_{end})$ has an inclusive start time $I_{start}$ and an exclusive end time $I_{end}$. The length of the interval is $L = I_{end} - I_{start}$.
The projections of an interaction set $\var{IS}$ ($\CIS$ or $\IIS$) to such a time interval are defined as functions with time interval parameters.
\begin{itemize}
    \item\emph{Touching}: $\mathit{T[I_{start}, I_{end})( IS ) = \set{int\in IS}{start(int) < I_{end} \wedge end(int)\ge I_{start}}}$
    \item\emph{StartingIn}: $\mathit{SI[I_{start}, I_{end})( IS ) = \set{int\in IS}{I_{start}\le start(int) < I_{end}}}$
    \item\emph{EndingIn}: $\mathit{EI[I_{start}, I_{end})( IS ) = \set{int\in IS}{I_{start}\le end(int) < I_{end}}}$
    \item\emph{ContainedIn}: $\mathit{CI[I_{start}, I_{end})( IS ) = \set{int\in IS}{I_{start}\le start(int) \le end(int) < I_{end}}}$
\end{itemize}
The actual overlap of a \emph{Touching} interaction $int \in \var{IS}$ with the time interval is between $\mathit{l\_limit = \max\{start(int), I_s\}}$ to $\mathit{r\_limit =  \min\{end(int), I_e\}}$, so $$\mathit{overlap[I_s,I_e)( int ) = \begin{cases} r\_limit - l\_limit & \text{if } start(int) < I_e \wedge end(int) \ge I_s  \\
0 & \text{otherwise}\end{cases}}$$
\end{definition}
Figure \ref{fig:interactionoverlap} shows some example interactions. The complete ones have arrowheads giving the $start$ and $end$ times and the incomplete ones are represented by \texttimes. All four complete ones are \emph{Touching}. The second one is also \emph{StartingIn}, the third one \emph{EndingIn} and the last one is both of those which makes it \emph{ContainedIn}. The top incomplete interaction is also everything while the bottom one is not touching at all. The overlap is colored red in the figure.

Finally, we define metrics for conformance, performance and process context.
For conformance, we propose two similar local fitness measures. One is based on the ratio of complete interactions and all interactions, while the other counts events instead of whole interactions. Since complete interactions are a pair of two events, the latter will always be higher, but it considers the actual time of occurrence more exactly.
\begin{definition}[Local fitness]
Let $\var{(CIS, IIS)}$ be a pair of sets of complete and incomplete interactions and $[I_s, I_e)$ be a time interval. The interactions starting in the interval are $\mathit{c\_start = SI[I_s, I_e)(\CIS)}$ and $\mathit{i\_start = SI[I_s, I_e)(IIS)}$ for complete and incomplete ones respectively.
$$\mathit{lfitness_{int}[I_s, I_e)( \CIS, \IIS ) = \begin{cases} \frac{|c\_start|}{|c\_start| + |i\_start|} & \text{if } |c\_start| + |i\_start| \neq 0\\
\text{undefined} & \text{otherwise}
\end{cases}}$$
The events from interactions inside the interval are $\var{ce\_start} = \set{e}{I_s \le time(e) < I_e \wedge e \in \set{e,e^\prime}{((e,t),(e^\prime,t^\prime))\in \CIS}}$ and $\var{ie\_start} = \set{e}{I_s \le time(e) < I_e \wedge ((e,t)) \in \IIS}$ for complete and incomplete interactions respectively.
$$\mathit{lfitness_{event}[I_s, I_e)( \CIS, \IIS ) = \begin{cases} \frac{|ce\_start|}{|ce\_start| + |ie\_start|} & \text{if } |ce\_start| + |ie\_start| \neq 0\\
\text{undefined} & \text{otherwise}
\end{cases}}$$
\end{definition}
If no interactions/events occur in the chosen time interval, the fitness is undefined. For the interval in Figure \ref{fig:interactionoverlap}, $\func{lfitness_{int}[I_{start}, I_{end})} = 0.67$ and $\func{lfitness_{event}[I_{start}, I_{end})} = 0.8$.

The current average sojourn time on a place is a measure for performance. The average is computed over complete interactions starting in the chosen interval. That means it is also undefined during intervals where no interactions occurred.
\begin{definition}[Local performance]
Let $\CIS$ be a set of complete interactions and $[I_s, I_e)$ be a time interval. Again, let $\mathit{c\_start = SI[I_s, I_e)(\CIS)}$.
$$\mathit{lperf[I_s,I_e)( \CIS ) = \begin{cases} \sum_{ci\in c\_start} duration(ci) / |c\_start| & \text{if } |c\_start|\neq 0\\
\text{undefined} & \text{otherwise} \end{cases}}$$
\end{definition}
$\func{lperf}$ describes how long (on average) cases that arrived at the place in the chosen interval eventually waited on there.

We propose three different metrics to try to measure the process context in terms of busyness at the current place during the chosen time interval. If many cases are currently waiting on this place, it will be busy.
\begin{definition}[Local busyness]
Let $\CIS$ be a set of complete interactions and $[I_s,I_e)$ be a time interval.
\begin{itemize}
    \item $\mathit{lbusyness_{c\_int}[I_s,I_e)( \CIS) = |SI[I_s,I_e)(CIS)|}$
    \item $\mathit{lbusyness_{activity}[I_s,I_e)( \CIS ) = \begin{cases} \sum_{ci \in CIS} overlap[I_s,I_e)(ci)/(I_e - I_s) & \text{if } I_s \neq I_e\\
    \text{undefined} & \text{otherwise} \end{cases}}$
    \item $\mathit{lbusyness_{remsojourn}[I_s,I_e)( \CIS ) = \sum_{ci\in T[I_s,I_e)(CIS)} (end(ci) - \max(start(ci),I_s))}$
\end{itemize}
\end{definition}

$\func{lbusyness_{c\_int}}$ is the number of complete interactions starting in the interval. Incomplete interactions may also inhibit performance since they are also in the log and actually occurred, so a variant $\func{lbusyness_{int}}$ counting all interactions may also be useful.
$\func{lbusyness_{activity}}$ is the sum of the ratios of overlap and the time interval size. The metric is relative to the interval size, so the interactions are weighed by their overlap which provides are more nuanced view on the busyness on a place during an interval. This is undefined for a time interval with the same start and end time.
$\func{lbusyness_{remsojourn}}$ is the total remaining sojourn time from the start of the interval of all touching interactions, shown in green in Figure \ref{fig:interactionoverlap}. In other words, if the place was a queue and could only handle one interaction after the other, it would take $\func{lbusyness_{remsojourn}}$ long to empty the queue. Instead of the total time, the average might also be interesting.

\section{Tying it all together} \label{alltogether}
The previous sections described the necessary process to arrive at metrics for conformance, performance and busyness. Given a system net $SN = (PN, M_i,M_f)$ and log $\Log$, the set of complete $\CIS$ and incomplete $\IIS$ interactions can be extracted for each individual place $p\in P$.
The metrics can then be calculated over time by dividing the whole process duration into a selected number of intervals or intervals of certain length like day, month or year. This yields a timeseries which can be used in statistical analysis to detect seasonality and trends.
By computing the relative standard deviation (standard deviation divided by mean) over the series or using other deviation metrics, one can also try to judge the stability of a certain metric over time. This way places can also be compared. Some might be in parts of the process that are quite streamlined and stable, while elsewhere the actual process execution may be more erratic in terms of conformance and performance. Figure \ref{fig:timeseriesschematic} shows a schematic of this on the running example model.
\begin{figure}
    \centering
    \includegraphics[width = \textwidth]{figures/concept/tyingtogether.pdf}
    \caption{Schematic result of applying the approach}
    \label{fig:timeseriesschematic}
\end{figure}
Another interesting application is finding correlations between these metrics. For example, does a high busyness value actually correlate with long sojourn times? In other words, how sensitive is this place to busyness? Or does a low conformance possibly lead to longer sojourn times? Since the $\CIS$ and $\IIS$ sets also contain the event and case data information, correlations can be extended to include those features. Which activity is often involved in incomplete interactions, and during which times? Do case attributes influence performance on this place? The localization of these questions to individual places and time frames makes it possible to find correlations which would be too weak to be detected over the whole log and process. For these purposes a dataset can be compiled containing all interactions with time and data information together with the metrics calculated for the duration of the interaction.

We also include another interesting aspect in our approach which is using the relative time since the case start. Every previously mentioned analysis can be performed for a relative time perspective by adjusting the intervals used for the metrics. It can be used to see after which time cases usually arrive at a place and if there are differences when they are later or earlier.

\chapter{Implementation} \label{chap:impl}
%%This chapter is used to show our implementation in ProM.  It can be split into 3 parts. 
%The first one is the Dfg method, including the weight update, process tree generation and petri net without long-term dependency generation.
% The second part is to add long-term dependency, it can be use as whole part or customized part into model, also removing the long-term dependency
% Evaluation part is the confusion matrix measurement.
%% Change implementation structure in this way::
%% 1. platform introduction, ProM + KNIME platforms
%% 2. No neccessary to describe the input right?? Aslo, there are another new concepts shown in the graph, which we need to avoid it. 
%% 3. then the screenshots to show the steps of the implementation
%% 4. If we introduce the property here, really, it will not help.. In this way, it can be fine.. But just add the introduction part for the KNIME
In this chapter, we begin with the introduction of implementation platforms for our methods and then show the use of those applications step by step.

\section{Implementation Platforms}
\subsection{Process Mining Platform -- ProM}
ProM is an open-source process mining tool in Java that is extensible by adding a set of plug-ins \cite{ProM}. ProM supports a wide variety of process mining techniques and is usually used for academic research. We implement the algorithm on ProM 6.8, which is the latest stable version. The corresponding plugin is \textbf{\emph{Repair Model By Kefang}} and released online \cite{MyPlugin}.

\subsection{KNIME}
KNIME Analytics Platform is an open-source software to help researchers analyze data. Multiple modules are integrated into this platform for loading, transforming and processing data. Researchers can achieve their goals by creating visual workflows composed of expected modules implemented as nodes with an intuitive, drag and drop style graphical interface, rather than focusing on any particular application area.

The reasons to integrate our techniques into KNIME are (1)KNIME is widely used in scientific research and benefits the application of our techniques;(2)KNIME supports automation of test workflow, which helps conduct more efficient experiments.  However, the integration requires additional development effort.
% here we need to change the name of our sections, because if we present them into a general way, so we need to show them in a general methods.
\section{Generate a Petri net}
Firstly two dialogs are poped up to set the arguments, such as the event classifier to  generate directly-follows graphs from event logs. Subsequently, a dialog is shown to set the Inductive Miner parameters. The parameters include the Inductive Miner variant and the noise threshold to filter the data. The dialog is displayed in Figure \ref{fig:dfg-IM-setting}.
\begin{figure}
	\centering
	\includegraphics[scale=0.75]{figures/implementation/dfg-IM-setting.png}
	\caption{Inductive Miner Parameter Setting}
	\label{fig:dfg-IM-setting}
\end{figure} \\
After setting the parameters, process models  of process tree and Petri net without long-term dependency can be generated by Inductive Miner and displayed in the result view in Figure \ref{fig:dfg-IM-pn-without-lt}. 
\begin{figure}
	\centering
	\includegraphics[width=\textwidth]{figures/implementation/dfg-IM-pn-without-lt.png}
	\caption{Generated Petri net without long-term dependency}
	\label{fig:dfg-IM-pn-without-lt}
\end{figure}
The left side is the model display area, where the right panel is to set the control parameters for the existing model, positive or negative instances. In interactive way, more flexibility is allowed by this plug-in to repair model. By default, the generated model type and the weight sliders are enabled at first. The control panel for adding long-term dependency are only triggered after choosing the option to repair model with long-term dependency. 

The model type is selected in the blue rectangle marked in Figure \ref{fig:dfg-IM-pn-without-lt}. It has 4 options to control the generated model type. Currently, the option "Show Petri net" is chosen, so the constructed model is Petri net without long-term dependency. The weights sliders are in red rectangle. They adjust the weights for the existing model, positive and negative instances. Once those options are submitted, different process models are mined under different weights. The rectangle in orange are the invisible part to control long-term dependency options. It will be discussed in the next section.

\section{Post Process to Add Long-term Dependency }
If we want to repair the Petri net with long-term dependency, one post procedure is triggered to add long-term dependency . This program in the background detects and puts places and silent transitions on Petri net directly mined from Inductive Miner to add long-term dependency. As comparison, the same weight setting is kept like the Figure \ref{fig:dfg-IM-pn-without-lt}, but the option to show a Petri net with long-term dependency is chosen. The resulted model is displayed in  Figure \ref{fig:dfg-IM-pn-with-lt}. 
\begin{figure}
	\centering
	\includegraphics[width=\textwidth]{figures/implementation/dfg-IM-pn-with-lt.png}
	\caption{Petri Net with long-term dependency }
	\label{fig:dfg-IM-pn-with-lt}
\end{figure}

Meanwhile, the control part of adding long-term dependency turns visible in the orange rectangle like in Figure \ref{fig:dfg-IM-pn-with-lt}.  It has two main options, one is to consider all long-term dependencies existing in the model, the other is to choose the part manually. It allows more flexibility for users. Below those two options are the manual selection panels, including a control part to add and remove pair. As an example, the blocks Xor(S1,S2) and Xor(T1,T2) are chosen to add long-term dependency. It results in the model in Figure \ref{fig:dfg-IM-pn-with-lt-m}. 
\begin{figure}[h]
	\centering
	\includegraphics[width=\textwidth]{figures/implementation/dfg-IM-pn-with-lt-manual.png}
	\caption{Petri net with selected long-term dependency}
	\label{fig:dfg-IM-pn-with-lt-m}
\end{figure}
\section{Post Process to Reduce Redundant Silent Transitions and Places}
By choosing the option of \emph{Petri net with LT After Reducing} in model type, silent transitions and places are reduced to simplify the model.
Under the same setting in Figure \ref{fig:dfg-IM-pn-without-lt}, the simpler model in Figure \ref{fig:dfg-IM-pn-with-lt-r} is constructed, after the post processing of reducing silent transitions.
\begin{figure}[h]
	\centering
	\includegraphics[width=\textwidth]{figures/implementation/dfg-IM-pn-with-lt-reduced.png}
	\caption{Petri net after reducing the silent transitions}
	\label{fig:dfg-IM-pn-with-lt-r}
\end{figure}

\section{Additional Feature to Show Evaluation Result}
Another feature in this plugin  is to show the evaluation result based on confusion matrix. With the brief evaluation result, it helps set the parameters for the optimal Petri net. 

After creating the current model in the left view, the evaluation program in background uses the event log and the current Petri net in the view as inputs. A naive fitness checking is applied on the repaired model with the event log. This procedure is based on the existing plugin in ProM -- \textbf{PNetReplayer}. This plugin checks if the trace fits the model and give out the one possible deviation with minimal cost. In our implementation, either the deviation on model or in trace is set with the same cost. Based on the deviation result and the label information on each trace, a confusion matrix is generated. Moreover, relative measurements like recall, precision are calculated and shown in the bottom of the left view in Figure \ref{fig:dfg-IM-cm}.  If the button of green rectangle in the right view \emph{Show Confusion Matrix} is pressed again, the program is triggered again and generates a new  confusion matrix result in the dark green dashed rectangle which will be listed above the previous result in light green dashes area. 
\begin{figure}
	\centering
	\includegraphics[width=\textwidth]{figures/implementation/dfg-IM-confusionmatrix.png}
	\caption{Generated Process Tree Model}
	\label{fig:dfg-IM-cm}
\end{figure}

\section{Integration into KNIME}
% this section describes the integration of our algorithm with KNIME, should we introduce some parts abotu them?? Yes, here about our real implementation, above should introduce the basic implementation steps.
\emph{Nodes} in the workflow represents different modules corresponding the plugins in ProM. Each node has certain input ports on the left side to represent the required parameters and  ports on the right to output result. By connecting the ports between nodes, data are passed and processed by one node after another. To integrate our algorithm into KNIME, other related modules on process mining are necessary, which can be divided into the following categories: 
% make a lot of work here to express your work
\begin{enumerate}
	\item Data importer and exporter. The importers and exporters for event logs, process trees and Petri nets are implemented to load and save basic data for Process Mining.
	\item Event logs manipulation. Nodes for splitting, sampling and assigning labels to event logs are implemented to benefit our experiments.
	\item Classic discovery algorithms. Inductive Miner and Alpha Miner are integrated into KNIME to provide baselines for our algorithm.
	\item Model enhancement. Our proposed method is integrated in KNIME to repair model in Petri net.
\end{enumerate}

To integrate our repair algorithm from ProM into KNIME, we need to create the workflow in the Figure \ref{fig:impl-KNIME}. After reading a Petri net by  \emph{PetrinetReader} and an event log by \emph{Import Event Log(XES)}, Node \emph{IncorporateNegInfo} applies the algorithm in this thesis to repair a model in Petri net with incorporating negative information. The outputs have different kinds of Petri nets to match the ones generated in ProM, eg. reduced Petri net with long-term dependency, Petri net without long-term dependency. In addition, we can evaluate our repaired model by using the node \emph{RepairEvaluator}. At last, we can save the repaired Petri net by \emph{PetrinetWriter}.
% give a screen shot and list the explaination on it.
\begin{figure}
	\centering
	\includegraphics[width=\textwidth]{figures/implementation/implementation-KNIME.png}
	\caption{Integration of our repair techniques into KNIME}
	\label{fig:impl-KNIME}
\end{figure}


\chapter{Evaluation} \label{chap:eval}
%%This is the evaluation part, int includes the following parts.
% <1> Evaluation Metrics, explain the measurements chosen for this experiments
% <2> Test Platform in KNIME, introduced before so we don't really need to repeat it here
% Firstly, we validate our methods and make sure it works for the properties.. Then using the whole data to test the weights tend and also, if it handled the real life data. 
% <3.1> validation part to check the methods work for those situations, but not necessary;; Then big data test to show property to handle those situations. 
% <3> Test Cases Design, the parameters we want to compare, the cases
%   ==> synthetic data
%   ==> Real life data
%   ++ Data to test the property
In this chapter, we evaluate the proposed repair techniques based on the quality of repaired model. At first, we define the evaluation criteria. Next, we briefly introduce the test platforms KNIME and relevant ProM plugins tools. Then, we conduct two kinds of tests. One is based one the demo example proposed in the introduction part, and the other is on the real life data. 
\section{Evaluation Criteria}
% First talk about our data and our model, then choose the confusion matrix as one measurements. But we should review the traditional measuremtns on process mining before introducing the confusion matrix. but we should also focus on the accuracy part and f-score.
We evaluate the repair techniques based on the quality of repaired models with respect to the given event logs. In process mining, there are four quality dimensions generally used to compare the process models with event logs. 
\begin{itemize}
	\item \emph{Fitness.} It quantifies the extent how well the model reproduces traces in the event log which is used to build the model.   
	\item \emph{Precision.} It quantifies the extent how the discovered model limits the completely unrelated behavior that doesn't show in the event log. 
	\item \emph{Generalization.} It addresses the over-fitting problem when a model strictly matches to only seen behavior but is unable to generalize the example behavior seen in the event log. 
	\item \emph{Simplicity.} This dimension captures the model complexity. According to Occam's razor principle, the model should be as simple as possible.
\end{itemize}
% How to come to confusion matrix?? 
The four traditional quality criteria are proposed in the environment where only positive instances are available. Therefore, when it comes to the model performance, where negative instances are also possible, the measurement metrics need to be adjusted. 

With labeled traces in the event log, the repaired model can be seen as a binary prediction model where the positive instances are supported while the negative ones are rejected. Consequently, the model evaluation becomes a classifier evaluation and confusion matrix is applied in our experiments.

% Describe its features and some derived measurements. 
Confusion matrix has a long history to evaluate the performance of a  classification model. A confusion matrix is a table with columns to describe the prediction model and rows for actual classification on data.  As seen as a binary classifier, the repaired model produces four outcomes according to confusion matrix -- true positive, true negative, false positive and false negative, which is shown in the Table \ref{tab:cm}.
\begin{itemize}
	\item True Positive(TP): The execution allowed by the process model has a positive performance outcome.
	\item False Positive(FP): The execution allowed by the process model has a negative performance outcome.
	\item True Negative(TN): The negative instance is blocked by the process model.
	\item False Negative(FN):The negative instance is enabled by the process model.
\end{itemize} 
% confusion matrix
\begin{table}[]
	\caption{Confusion Matrix}
	\label{tab:cm}
	\begin{tabular}{ll|c|c|}
		\cline{3-4}
		&                   & \multicolumn{2}{c|}{repaired model}                                               \\ \cline{2-4} 
		\multicolumn{1}{l|}{}                                                                         &                   & \multicolumn{1}{l|}{allowed behavior} & \multicolumn{1}{l|}{not allowed behavior} \\ \hline
		\multicolumn{1}{|l|}{\multirow{2}{*}{\begin{tabular}[c]{@{}l@{}}actual \\ data\end{tabular}}} & positive instance & TP                                    & FN                                        \\ \cline{2-4} 
		\multicolumn{1}{|l|}{}                                                                        & negative instance & FP                                    & TN                                        \\ \hline
	\end{tabular}
\end{table}
Various measurements can be derived from confusion matrix. According to our application, the following criteria are chosen. Generally, there is a trade-off between the quality criteria. So the measurements below are only used to evaluate specific aspects of repair techniques.
\begin{itemize}
	\item Recall. It represents the true positive rate and is calculated as the number of correct positive predictions divided by the total number of positives.
	\[Recall = \frac{TP}{TP + FN}\]
	\item Precision. It describes the ability of the repaired model to produce positive instances.
	\[Precision = \frac{TP}{TP + FP }\]
	%\item specificity. In opposite with recall, it measures the true negative rate.
	%\[Specificity = \frac{TN}{TN + FP}\]
	\item Accuracy. It is the proportion of true result among the total number. It  measures in our case how well a model correctly allows the positive instances or disallows the negative instances.
	\[Accuracy = \frac{TP+TN}{TP+TN+FP+FN}\]
	\item F-score is is the harmonic mean of precision and recall.
	\[F_1 = \frac{2*Recall*Precision}{Precision + Recall}\]
\end{itemize}

\section{Experiment Platforms}
KNIME, as a scientific workflow analytic platform, supports automation of test workflow, which helps us repeat experiments efficiently. Yet, the integration of traditional process mining plugins into KNIME is out of our capability due to the time limit. Therefore, partial experiments with current repair techniques are still conducted in ProM. 
\subsection{KNIME}
% this section describes how KNIME supports automatic test, FlowVariable and optimization parts of it.
KNIME supports automation of test workflow mainly through the following mechanisms. 
\begin{itemize}
	\item Loop Control Structure. KNIME provides a bunch of control nodes which support re-executing workflow parts.  Two nodes \emph{Loop Start} and \emph{Loop End} explicitly express the beginning and end of a loop structure, where the workflow between those two node is the loop body and is executed recursively in a fixed number, or until certain conditions are met. In our test, we repeat our repair techniques for different parameter settings by applying loop structure into KNIME workflow.
	\item Flow Variables. Flow Variables are used inside a KNIME workflow to parameterize node settings dynamically. When it combines with loop control structure, tests with different settings is able to conduct automatically.
\end{itemize}
Furthermore, there are nodes provided by KNIME to optimize the value of some parameters with respect to a cost function. As long as the cost function is provided, KNIME is able to automatically optimize the corresponding parameters. 

\subsection{ Experiments with ProM Plugins}
Due to the frequent errors on the corresponding plugin, we exclude the tests on repair techniques in \cite{dees2017enhancing} and conduct experiments with the following types. 
\begin{itemize}
	\item \textbf{Type 1} \textbf{Inductive Miner} only on the positive event log to discover a model. The default setting with infrequent variant and noise threshold as 20 is chosen. Later, the mined model is checked on the labeled event with positive and negative instances. This method is abbreviated as IM.
	\item \textbf{Type 2} \textbf{Repair Model} from \cite{fahland2015model} is applied on the positive event log to discover a model. The default setting is chosen. Later, the mined model is checked on the labeled event with positive and negative instances. This method is abbreviated as Fahland, named after the name of main author.
	\item \textbf{Type 3} \textbf{Dfg-Repair from our thesis} is applied on the labeled event log with positive and negative instances. Default setting  for the control parameters is 1.0 while the parameters to generate Petri nets from directly-follows graph are set as the same as experiment Type 1.  Later, the repaired model is evaluated on the labeled data. 
\end{itemize}

\section{Experiment Results}
We conduct our experiments into two main parts. One is to verify if our method overcomes the limits of current repair algorithms. This experiment is based on the synthetic data and models from Introduction chapter. The other experiment is based on real life data, in order to test the feasibility of our repair techniques.
%For convenience,we refer the repair methods in \cite{fahland2015model} by the main auther's name Fahland, and the repair techniques in \cite{dees2017enhancing} as Dees. The Inductive Miner is abbreviated as IM. Our method which built on directly-follows graph is denoted as Dfg-repair. 
 
\subsection{Test on Demo Example}
In this part, experiments are performed on the motivating examples which are listed in Introduction. Thereby, we are able to answer whether our repair method overcomes shortcomings of current techniques which are shown in the introduction chapter. 
\subsubsection{Answer to Situation 1}
\emph{Situation 1} shows the drawbacks of current repair methods \cite{fahland2015model, dees2017enhancing} that unexpected behaviors are introduced into model by adding subprocess in the form of loops. Moreover, rediscover strategy with IM doesn't take the original model into account and generates a new model that deviates from the original model. 


Given the process model $M_0$ and the event log $L_1$, additional activities \textbf{x1,x2} in $L_1$ lead to good performance and need to be added into the model $M_0$. Applying proposed repair techniques dfg-repair, we obtain the  repaired model listed in Figure \ref{fig:demo_dfg_s1}. The parameters for our method are set in the following : weight for the existing model is 0.45, weight for positive examples is 1.0, the Inductive Miner for Infrequent is chosen and has a noise with 20, which is the same setting as the rediscovery method by Inductive Miner in Introduction. 

As seen in Figure \ref{fig:demo_dfg_s1}, the subprocesses for \textbf{x1,x2} are added in a sequence with others. In this way, $M_{1.3}$ is able to reflect the deviations in positive instances while keeping similar to the reference model $M_0$. Compared to techniques in \cite{fahland2015model}, it increases the precision without loops.
% we should give our confusion matrix result
\begin{figure}[htp]
	\centering
	\begin{subfigure}[b]{0.31\textwidth}
		\centering
		\includegraphics[width=0.75\linewidth, height=0.8\textheight]{figures/evaluation/PN-result-demo-s1-dfg.pdf}
		\caption{$M_{1.3}$ for situation 1}
		\label{fig:demo_dfg_s1}
	\end{subfigure}%
\quad
	\begin{subfigure}[b]{0.31\textwidth}
		\centering
		\includegraphics[ width=0.75\linewidth, height=0.8\textheight]{figures/evaluation/PN-result-demo-s2-dfg.pdf}
		\caption{$M_{2.3}$ for situation 2}
		\label{fig:demo_dfg_s2}
	\end{subfigure}
 \quad
\begin{subfigure}[b]{0.32\textwidth}
	\centering
	\includegraphics[ width=0.75\linewidth, height=0.8\textheight]{figures/evaluation/PN-result-demo-s3-dfg.pdf}
	\caption{ $M_{3.3}$ for situation 3}
	\label{fig:demo_dfg_s3}
\end{subfigure}
	\caption{repaired models with our techniques for situation 1,2 and 3 in Introduction part. The green place is the initial marking of the Petri net and the doubled place is the final marking.}
	\label{fig:demo_dfg}
\end{figure}
\subsubsection{Answer to Situation 2}
Situation 2 describes the inability of current repair methods that fitting traces with negative performance outcomes cannot be used to repair a model. 
% some explaination about the situation.
The execution order of \textbf{e1, e2} affects the performance outcomes and \textbf{e1} is expected to position before \textbf{e2}. Without negative information, the repaired models have the same structure as the reference ones, because the execution of \textbf{e2} before \textbf{e1} brings also the positive outcomes. 

If we apply our repair methods on the model $M_0$ and event log $L_2$, with 1.0 for all control weights, and the same Inductive Miner-Infrequent with noise 20, the repaired model $M_{2.3}$ is obtained. In $M_{2.3}$, \textbf{e1} is executed before \textbf{e2}. It shows that our method is able to incorporate the negative information and balance the forces from the existing model, positive and negative instances. 

\subsubsection{Answer to Situation 3}
% Conclusion part
Situation 3 concerns the long-term dependency in Petri nets, which is not handled in current repair and rediscovery techniques. As observed in event log, there exists the long-term dependency set, $LT=\{ a1\rightsquigarrow d1, a2\rightsquigarrow d2\}$.  With adding long-term dependency as expected in Figure \ref{fig:demo_s3_expected}, precision and accuracy increase, since the model limits activity selection and blocks the negative behavior due to free execution of xor branches. Yet none of the current repair  and rediscovery techniques are able to detect and add long-term dependencies in the Petri net. 

In our repair techniques, the long-term dependency is taken into account with negative information. With inputs of the Petri net $M_0$ and event log $L_3$, our methods produces the repaired model $M_{3.3}$ with long-term dependency. Two silent transitions that are used to explicitly represent the long-term dependencies can be deleted with post procedure to reduce the redundant silent transitions and places. After reduction, our repaired model is simplified as the model $M_{3}$. 
\subsubsection{Comparison with Confusion Matrix}
In this section, we list the evaluation results of the repaired models based on confusion matrix. In Table \ref{tab:demo-result}, for Situation 1 with only positive instances, the repair techniques give the same confusion matrix result. However,  $M_{1.2}$  with loops implicates a lower precision. In Situation 2, with current techniques or rediscovery methods in IM, the model stays the same as the reference  model $M_0$. Since no negative instance is rejected, the recall is 1 but precision is below 0.6.  In comparison, dfg-repair uses the negative instances and adjusts the model correspondingly. Therefore, the repaired model $M_{2.3}$ has higher precision, accuracy and F1 score. In Situation 3 with long-term dependency, our method succeeds to detect and add the  long-term dependency in the model. In this way, no false positive or false negative instances are in the confusion matrix, and the repaired model holds the highest values for all listed measurements.
\begin{table}[h]
	\centering
	\caption{Test Result on BPI15-M1 data}
	\label{tab:demo-result}
	\resizebox{\textwidth}{!}{
		\begin{tabular}{lll|llllllll|}
			\hline
			\multirow{2}{*}{\thead{Situation}} & \multirow{2}{*}{\thead{method} }                &    \multirow{2}{*}{\thead{Generated \\ model}}       & \multicolumn{8}{l|}{ \thead{confusion matrix metrics}}                                \\
			\cline{4-11}
			&  &     &
			\thead{TP}  & \thead{FP} & \thead{TN}  & \thead{FN}  & \thead{recall} & \thead{precision} & \thead{accuracy} & \thead{F1}             \\
			\hline
			S1      & IM              & $M_{1.1}$ & 50 & 50 & 0 & 0 & 1   & 0.5     & 0.5     & 0.667               \\
			S1     & Fahland              & $M_{1.2}$   & 50   & 50  &0     &0     & 1   & 0.5     & 0.5     & 0.667               \\
			
			S1      & Dfg-repair              &    $M_{1.3}$    &  50   &  50  &  0  &    0 &  1   & 0.5     & 0.5     & 0.667               \\
			
			\hline
			S2      & IM/Fahland              & $M_0$   & 60    & 45   &  0  & 0    &  1     &    0.571       & 0.571         &  0.727   \\
			
			S2      & Dfg-repair              & $M_{2.3}$      & 50    &   5 &   40  & 10    &  0.833      &   0.909        &  0.857         &  0.870      \\
			
			\hline
			S3     & IM/Fahland             & $M_{0}$   & 100    & 100   &  0   &  0   &  1.0      &   0.5        &   0.5       &  0.667    \\
			
			S3      & Dfg-repair             & $M_{3.3}$       &  100   &   0 &  100   &  0   &  1      &   1        &  1        &   1     \\
			\hline         
	\end{tabular}}
\end{table}

In conclusion, our proposed method is able to overcome shortcomings of current techniques mentioned in the Introduction. It avoids the loops in model by repairing the model with additional activities, incorporates the negative information in the data to adjust the model, also detect and add the long-term dependency into model. In this way, the repaired model has better  recall, and accuracy.

\subsection{Test on Real Life Data}
% here we will list all the data here but before describe the test data
We choose publicly available event logs from BPI challenge 2015 and build a data set from them to test the feasibility of proposed repair techniques.
\subsubsection{Data Description}
The data set for BPI Challenge 2015 contain 5 event logs which are provided by five Dutch municipalities respectively. Those event logs describe the building permit application around four years. We choose it as our user cases due to the following reasons.
\begin{itemize}
	\item The event logs hold attributes as potential KPIs to classify traces. Attribute \textbf{SUMleges} which records the cost of the application is a candidate to label traces as positive or negative if its value  is over the threshold. What's more, we can take the throughput time of the application as another potential KPI. \\
	In a word, this data set provides us information to reasonably label traces.
	\item The five event logs describe an identical process, but includes deviations caused by the different procedures, regulations in those municipalities. Also, the underlying processes have changes over four years.\\
	So, this data set gives us a basic process but also allows deviations of the actual event logs and predefined process, which builds the environment for repair techniques.
\end{itemize}
We conduct our experiments on those event logs. However, due to the time limits, we only managed to get the result on experiments with the event log \textbf{BPIC15\_1.xes.xml}. This event log includes 1199 cases and 52217 events in 398 classes. We preprocess the event log and get a proper subset of data as our user case.  
\begin{table}[h]	
	\caption{Test event log from real life data BPI15-1}
	\label{tab:event-log}
	\begin{tabular}{|l|l|l|l|l|}
		\hline
		Data ID & Data Description                                & Traces Num & Events Num & Event Classes \\ \hline
		D1      & \makecell{Heuristic filter  \\ with 40 }                     & 495        & 9565       & 20             \\ \hline
		D2      & \makecell{Apply heuristic filter \\ on D1 with 60      }     & 378        & 4566       & 12            \\ \hline
		D3.1    & \makecell{classify on SumLedges;  \\ values below 0.7 as positive} & 349        & 6744       & 20             \\ \hline
		D3.2    & \makecell{classify on SumLedges;  \\ values above 0.7 as negative }& 146        & 2811       & 20             \\  \hline
		D3.3    & union of D3.1 and D3.2                             & 495        & 9596       & 20             \\ \hline
		D4.1    & \makecell{ classify on throughput time;  \\ values below 0.7 as positive} & 349        & 6744       & 20             \\ \hline
		D4.2    & \makecell{classify on throughput time;  \\ values above 0.7 as negative} & 146        & 2811       & 20            \\ \hline
		D4.3    & union of D4.1 and D4.2                             & 495        & 9596       & 20           \\ \hline
	\end{tabular}
\end{table}

We filter the raw event log by \textbf{\emph{Filter Log By Simple Heuristic}} in ProM with the following setting. 40 for the start, end  activities and the events between them, at end. We get the event log $D1$. After this, we calculate the throughput time for each trace and add it as a trace attribute \textbf{throughput time}. 
Then we classify traces according to  \textbf{SUMleges} and  \textbf{throughput time} separately. When our performance goal is to reduce the cost of application, if \textbf{SUMleges} of one trace is over 0.7 of the whole traces, this trace is treated as negative, else as positive. The similar strategy is applied on the attribute \textbf{throughput time}. A trace with \textbf{throughput time} higher than 0.7 of all traces is considered as a negative instance. Following this preprocess, we have event logs in Table \ref{tab:event-log} available for our tests. 


\begin{table}[htp]
	\caption{Generated reference models for test}
	\label{tab:ref-models}
	\resizebox{\textwidth}{!}{
	\begin{tabular}{|llll|lllllllll|}
		\hline
		\multirow{2}{*}{\thead{Model\\ ID}} & \multirow{2}{*}{\thead{Used \\Data}} & \multirow{2}{*}{\thead{Setting}}  &  
		\multirow{2}{*}{\thead{Event\\Class}} & \multicolumn{8}{c|}{\thead{CM Evaluation}}                     \\ 
		\cline{5-13}
		&                                                                           &                                                                             &                                                                            &Data & TP & FP & TN & FN & recall & precision & accuracy & F1 \\ \hline
		M1                                                                      & D1                                                                        & \makecell[l]{IM-infrequent: \\ Noise Setting: 20} & 20                                                                         & D3.3 & 112   & 40   & 106   & 237   & 0.321       &0.737           &   0.440       &   0.447 \\
		
		&                                                                      & &                                                                          & D4.3 & 131   &  21  &  128  & 215   & 0.379       & 0.862           & 0.523          & 0.526    \\
		
		\hline
		M2                                                                      & D1                                                                        & \makecell[l]{IM-infrequent: \\ Noise Setting: 50} & 20                                                                         & D3.3 & 106   & 39   &  107  & 243    & 0.304        &  0.731         &0.430          &0.429    \\
		
		&                                                                      & &                                                                          & D4.3 & 125   &   20 & 129   &221    &    0.361    & 0.862           & 0.513          &0.509    \\
		\hline
		
		M3                                                                      & D2                                                                        & \makecell[l]{IM-infrequent: \\ Noise Setting: 20} & 12                                                                         &D3.3 &  0  & 0   & 146   & 349   & 0       &NaN           &    0.295      & 0   \\
		
		&                                                                      & &                                                                          & D4.3 &  0  & 0   & 149   & 346   & 0       & NaN           &  0.301        &0    \\
		\hline 
		
		M4                                                                      & D2                                                                        & \makecell[l]{IM-infrequent: \\ Noise Setting: 50} & 12                                                                         &D3.3 &  0  &  0  &  146  & 349   & 0       & NaN           &     0.295     & 0\\   
		&                                                                      & &                                                                          & D4.3 &  0  & 0   & 149   & 346   & 0       & NaN           &  0.301        &0    \\
		
		\hline
	\end{tabular}
 }
\end{table}
Based on the filtered data, we derive corresponding Petri nets as reference process models. The Table \ref{tab:ref-models} lists the models with different setting. \textbf{IM-infrequent} is one variant of Inductive Miner working on event logs with infrequent traces. \textbf{Noise} is set as the threshold to filter out infrequent traces. After mining a reference model, we compare them with  corresponding event logs to get the basis lines for later evaluation.
% should we explain the data and the model, they are different with old files!!
% Please add the following required packages to your document preamble:
% \usepackage{multirow}

As seen in table above, the reference models don't apply well to the corresponding event logs. So changes on the models are in demand, to reflect better the reality and also to enforce the positive instances and avoid negative instances. 
\subsubsection{Test Result}
% we don't need to transfer page to landscape view, because we have many rows too.
Three types of repair techniques which are called \textbf{IM}, \textbf{Fahland} and \textbf{Dfg-repair} are applied on preprocessed event log set in Table \ref{tab:event-log} and models from Table \ref{tab:ref-models}. The experiment result is listed in the Table \ref{tab:rl-result}. For better understanding, we give the details of one experiment set which is conducted with the reference model M3 and event log D3.1 and D3.3.  

Figure \ref{fig:rl_ref} displays the reference model M3, which has 0 TP and 0 FP compared to D3.3. It implies that M3 leads to no positive performance outcomes. Firstly, Inductive Miner for infrequent traces with noise threshold 0.2 is used on the positive event log D3.1 to rediscover a model. The generated model is shown in Figure \ref{fig:rl_IM}, which has changed a lot compared to the original model M3 as shown in the Figure \ref{fig:rl_ref}. After getting the generated model, we compare it with labeled event log D3.3 and compute the confusion matrix criteria. As shown in Table \ref{tab:rl-result}, the repaired model tend to have high FN and low FP, because the 

 
Since IM doesn't take the reference models into account, it always output the same model based on the event log. 

After applying the Fahland's repair techniques from \cite{fahland2015model}, the reference model M3 is repaired as in Figure \ref{fig:rl_fahland}. With duplicated transitions, loops and silent transitions for adding subprocesses, it is more complicated than the reference model M3. When evaluating the repaired model with confusion matrix, we get 349 TP, 145 FP, 1 TN, and 0 FN, which leads to high recall. However, it needs to notice that the principle of Fahland repair techniques is to add subprocesses for deviations in the model. Without consideration of eliminating negative behavior, the repaired model is likely to damage the model precision and accuracy.

Dfg-repair techniques are firstly applied with the default setting, 1 for all control parameters. It results in a model with 0 TP, 0 FP, 146 TN, and 349 FN,  which contrasts the result by Fahland repair techniques. This is probable because the forces from the existing model and negative instances exceed the positive force, and blocks  behavior with positive outcomes. 

As a comparison, we change the control parameter setting to 0.5 for the existing model, 1 for the positive instances, 0.5 for the negative instances. After repeating the experiment, the confusion matrix changes to 131 TP, 63 FP, 83 TN, and 217 FN, while recall increases from 0 to 0.378, accuracy from 0.294 to 0.428, and F1 from 0 to 0.485. Apparently, the quality is improved due the the new setting. The possible reason is that three forces are balanced more properly for the repaired model.

% here to fix the evaluation from last step!! Better to give a good result on it!! Put it aheda before the whole result, and then organize it later.
\begin{figure}[htp]
	\centering
	\begin{subfigure}[b]{0.45\textwidth}
		\centering
		\includegraphics[width=0.7\textwidth, height=0.9\textheight]{figures/evaluation/BPI_1_40_M3_figure.pdf}
		\caption{reference model M3}
		\label{fig:rl_ref}
	\end{subfigure}%
\quad
\begin{subfigure}[b]{0.45\textwidth}
	\centering
\includegraphics[width=1.0\textwidth, height=0.9\textheight]{figures/evaluation/PN-result-D4-1-IM20-pos.pdf}
\caption{repaired model with IM}
\label{fig:rl_IM}
\end{subfigure}%
\caption{The left model is the reference model M3. The rediscovery algorithm IM generates a new model based on the positive instances which is shown in the right side.}
\end{figure}
%two figures to show the reference models and IM generated model
\begin{figure}[htp]
	\centering
	\begin{subfigure}[b]{0.55\textwidth}
		\centering
		\includegraphics[width=1.0\textwidth, height=0.9\textheight]{figures/evaluation/PN-result-D4-3-M3-fahland-new.pdf}
		\caption{repaired model with Fahland}
		\label{fig:rl_fahland}
	\end{subfigure}%
	\begin{subfigure}[b]{0.43\textwidth}
		\centering
		\includegraphics[width=.9\textwidth, height=0.9\textheight]{figures/evaluation/PN-result-D4-3-M3-dfg-05-1-05.pdf}
		\caption{repaired model with Dfg-repair}
		\label{fig:rl_dfg}
	\end{subfigure}
	\caption{Repaired model for $M_3$ based on event log D3.1 with positive instances for Fahland's repair techniques  while event log D3.3 with both positive and negative instances for Dfg-repair. The weight setting for Dfg-repair is 0.5 for the existing model, 1 for positive and 0.5 for negative instances. Default setting is chosen for Fahland's method.}
\end{figure}


\begin{table}[ht]
	\centering
	\caption{Test Result on BPI15-M1 data}
	\label{tab:rl-result}
	\resizebox{\textwidth}{!}{
		\begin{tabular}{lll|llllllll|}
			\hline
			\multirow{2}{*}{\thead{event \\ log}} & \multirow{2}{*}{\thead{reference \\ model} }                &    \multirow{2}{*}{\thead{method}}       & \multicolumn{8}{l|}{ \thead{confusion matrix metrics}}                                \\
			\cline{4-11}
			&  &     &
			\thead{TP}  & \thead{FP} & \thead{TN}  & \thead{FN}  & \thead{recall} & \thead{precision} & \thead{accuracy} & \thead{F1}             \\
			\hline
			D3.1      & -              & IM & 137 & 48 & 118 & 289 & 0.32   & 0.74      & 0.43     & 0.45                \\
			D3.1      & M1              & Fahland   & 343   & 136  &10     &6     &0.983        & 0.716           &   0.713       & 0.829         \\
			
			D3.3      & M1              & Dfg-repair:1-1-1       &  124   &  52  &  94   &    225 &   0.355     &     0.705      & 0.44         & 0.472            \\
			D3.3      & M1              & Dfg-repair:0.5-1-0.5       &  155   &  66  &  80   &    194 &   0.444     &     0.701     & 0.474         & 0.544            \\
			\hline
			D3.1      & M2              & Fahland   & 317    & 133   &   13  & 32    &  0.908      &    0.704       & 0.667         &  0.793   \\
			
			D3.3      & M2              & Dfg-repair:1-1-1       & 124    &   52 &   94  & 225    &  0.355      &   0.705        &  0.44         &  0.472      \\
			
			D3.3      & M2              & Dfg-repair:0.5-1-0.5       & 155    &   66 &   80  & 194    &  0.444      &   0.701        &  0.475         &  0.544      \\
			\hline
			D3.1      & M3              & Fahland   & 349    & 145   &  1   &  0   &  1.0      &   0.706        &   0.707       &  0.828    \\
			
			D3.3      & M3              & Dfg-repair:1-1-1       &  0   &   0 &  146   &  349   &  0      &   NaN        &  0.295       &   0      \\
			D3.3      & M3              & Dfg-repair:0.5-1-0.5       &  132   &   63 &  83   &  217   &  0.378      &   0.677        &  0.434        &   0.485      \\
			\hline
			
			D3.1      & M4              & Fahland   &  349   & 144   &  2   &  0   & 1.0       &   0.708        &   0.709       &  0.829         \\
			
			D3.3      & M4              & Dfg-repair:1-1-1       &  0   &  0  &   146  &  349   &  0     &      NaN     & 0.294        & 0   \\
			D3.3      & M4              & Dfg-repair:0.5-1-0.5       &  125   &  59  &   87  &  224   &  0.358     &      0.679     & 0.428        & 0.469   \\
			\hline
			D4.1      & -              & IM &  131   &  21  & 128    &  215   &    0.379    & 0.862           &   0.523       &    0.526     \\
			D4.1      & M1              & Fahland   &  325   &  133  &  16   & 21    &   0.939     &  0.710         &  0.689       & 0.808                \\
			
			D4.3      & M1              & Dfg-repair:1-1-1       &  139   & 36   & 113    &   207  &   0.402     &   0.794        &   0.509       &   0.534                   \\
			D4.3      & M1              & Dfg-repair:0.5-1-0.5       &  172   & 48   & 101    &   174  &   0.497     &   0.782       &   0.552       &   0.608                   \\
			\hline
			D4.1      & M2              & Fahland   & 325    &  130  & 19    &  21   & 0.939       &    0.714       &  0.695        &  0.811                \\
			
			D4.3      & M2              & Dfg-repair:1-1-1     &  139   & 36   & 113    &   207  &   0.402     &   0.794        &   0.509       &   0.534              \\
			D4.3      & M2              & Dfg-repair:0.5-1-0.5     &  172   & 48   & 101    &   174  &   0.497     &   0.782        &   0.552       &   0.608              \\
			\hline
			D4.1      & M3              & Fahland   & 87    &  29  & 120    &  259   & 0.251       &    0.75       &  0.418        &  0.377                    \\
			D4.3      & M3              & Dfg-repair:1-1-1       &  0   &  0  & 346    &  149   &  0      &   NaN        &   0.303       &  0       \\ 
			D4.3      & M3              & Dfg-repair:0.5-1-0.5       &  182   &  49  & 164    &  100   &  0.526      &   0.788        &   0.70       &  0.631      \\ 
			
			\hline
			D4.1      & M4              & Fahland   & 63    &  20  & 129    &  283   & 0.182       &    0.759       &  0.388        &  0.294                 \\
			
			D4.3      & M4              & Dfg-repair:1-1-1       &   0  & 0   &  346   & 149    &   0     &    NaN       &     0.303     &  0      \\  
			D4.3      & M4              & Dfg-repair:0.5-1-0.5       &   172  & 48   &  101   & 174    &   0.497     &    0.782       &     0.552     &  0.608      \\      
			\hline              
	\end{tabular}}
\end{table}
% add some analysis on them..

As the results shown in Table \ref{tab:rl-result}, Fahland's repair techniques from \cite{fahland2015model} tend to have high recall but low values for true and false negative. The possible reasons are that (1) it repairs the reference model with the positive instances, which addresses the fitness of positive traces. (2) it repairs the model by adding subprocesses and introduces more behavior into the model, which also allows for the negative instances. Inductive Miner rediscovers a new model from the given positive instances, while the reference models are simply ignored. As a result, the generated model in Figure \ref{fig:rl_IM} has changed a lot compared to the original model M3 as shown in the Figure \ref{fig:rl_ref}.  

Dfg-repair techniques uses control parameters to balance the forces from the existing model, positive and negative instances. Therefore, with different setting, Dfg-repair techniques repair the model in different ways. To address this phenomenon, besides the default setting with value 1 for all parameters, another setting with 0.5 for the existing model, 1 for the positive instances, 0.5 for the negative instances is used to conduct experiments. As shown in the Table \ref{tab:rl-result}, with different settings, the repaired models change. Compared to the default setting, the setting with values 0.5, 1 and 0.5 results in models with higher recall, accuracy, and F1 score. The reasons behind might be that the force from negative instances affects model a lot, with the weight 1.0. It possibly blocks the behavior which contributes to positive performances.  With lower value on it, the forces from the existing model, positive and negative are balanced better and the quality of the repaired model is improved. As an example, the experiments on D3.3 and M3, or D.3. and M4, which shows the quality changes due to different setting.

Except for the weight for negative instances, the weight for the existing model also affects the quality of repaired model. The influence of the weights on the repaired model is analyzed later in the section. 


Due to the different settings in our method, forces from the reference model, positive, and negative event logs are balanced differently during repair, which results in different process models. To investigate the effect of those setting on the repaired model, we repeat our experiments on the following setting. 

Each of three control parameters for the existing model, positive and negative instances changes value from 0.0 to 1.0 with step 0.1. With this setting, directly-follows relation is generated. Afterward, the default setting of Inductive Miner Infrequent with noise threshold 20 is used to mine Petri nets from the generated directly-follows graph. In total, thousand experiments are conducted to investigate Dfg-repair method. Based on those results, we draw plots to show the tendency of evaluation results on the parameters for the existing model, positive and negative event logs. 

\begin{figure}[htb]
	\includegraphics[width=\linewidth]{figures/evaluation/M3-D43-ext-weight-plot.pdf}
	\caption{result with control parameter for existing model on event log D3.3 and model M3}
	\label{fig:ext-weight}
\end{figure} 
From the Figure \ref{fig:ext-weight}, with the parameter for the existing model going up, recall goes up while accuracy and precision goes down. The reason behind it is possibly ????.


\begin{figure}[htb]
	\includegraphics[width=\linewidth]{figures/evaluation/M3-D43-neg-weight-plot.pdf}
	\caption{result with control parameter for negative instance on event log D3.3 and model M3}
	\label{fig:neg-weight}
\end{figure}
Figure \ref{fig:neg-weight} shows that if the parameter for negative event log increases, precision and accuracy go up. By addressing negative force, our techniques tend to block behavior which leads to low performance output. However, if the negative force is over the force from positive event log and the existing model, certain behavior which contributes to positive performance will also be deleted from the models. In contrast, this creates a model with less recall. 


\begin{figure}[htb]
	\includegraphics[width=\linewidth]{figures/evaluation/M3-D43-pos-weight-plot.pdf}
	\caption{result with control parameter for positive instance on event log D3.3 and model M3}
	\label{fig:pos-weight}
\end{figure}
Figure \ref{fig:pos-weight} displays the tendency with the parameter for positive event log. When the positive parameter rises, the recall increases. Precision and accuracy also increases but with ???.


\chapter{Conclusion} \label{chap:conclusion}
% what kind of conclusion do you want to get?? My method works in some cases while other methods can not. 
% <1> deal with some situations
% <2> improve precision and accuracy compared to existing model
% Further work:<1> consider multiple KPIs in the model and adjust them
%    <2> Add long-term dependency directly on Petri net
%    <3> deal with the unsound model and fix them in some way
%    <4> simply using the threshold to generate model, but want to integrate with machine learning and generate good model

% Also, some limitations about our methods, we need to point them out.
%% can not give out the best setting
%% it is absoulte values, not vary due to the problem.
In this thesis, we explore ways to use negative information in model repair and propose our innovative method. Firstly, we analyzed the current techniques of model repairs based on performance and detected their shortcomings. Then we  proposed a general framework to incorporate the forces from the existing model, positive and negative event logs. Three abstraction data models in the same type are built to represent those forces. Later, forces are balanced based on those data models and expressed in a new data model. From this new data model, process models are discovered and converted into repaired models with the required type. Optional post processes include long-term dependency detection and silent transition reduction, which further improve the repaired models. 

Moreover, we demonstrate the usage of our method by conducting experiments with synthetic data and real life data. In the situations shown with synthetic data,  our method is able to overcome the shortcomings of the current repair techniques and provide repaired models with higher accuracy and precision. In experiments with the real life data, it shows that the proposed techniques are capable to produce models which better quality measurements while keeping the similarity between the repaired and reference models. Additionally, with respect to other methods, it runs faster and generates simpler models. 

As future work, we consider improving the rules of balancing different forces, which choose the directly-follows relation on the simple subtraction and sum of those forces. Advanced data mining techniques such as association rules discovery, can be used on those forces to derive rules for building a process model. The same improvement can be applied on the long-term dependency discovery. Moreover, in this thesis, the long-term dependency discovery is restricted on the activities with exclusive choice relations. Later, we can extend the long-term dependencies on other possible relations. Also, we can discard the process tree as our intermediate result and detect long-term dependencies directly on the Petri net. 

In conclusion, our proposed repair method is able to incorporate negative instances with the existing model and positive instances, which provides an innovative way to repair models in Process Enhancement. 

\fi
%%% add ``References'' section to TOC
%\phantomsection

\bibliography{references} \addcontentsline{toc}{chapter}{Bibliography}

\end{document}